\subsection{Bài 9: Đếm số dãy tương tự dãy a mà có tích các số là
chẵn.}\label{buxe0i-9-ux111ux1ebfm-sux1ed1-duxe3y-tux1b0ux1a1ng-tux1ef1-duxe3y-a-muxe0-cuxf3-tuxedch-cuxe1c-sux1ed1-luxe0-chux1eb5n.}

\begin{itemize}
\tightlist
\item
  Dãy tương tự khi \(|b[i] - a[i]| <= 1\) với mọi \(i\).
\item
  \(b[i]\) có thể nhận một trong 3 giá trị:
  \(a[i] - 1, a[i], a[i] + 1\).
\item
  Sử dụng duyệt đệ quy để thử tất cả các trường hợp.
\end{itemize}

\begin{Shaded}
\begin{Highlighting}[]
\DataTypeTok{int}\NormalTok{ n}\OperatorTok{,}\NormalTok{ a}\OperatorTok{[}\DecValTok{11}\OperatorTok{],}\NormalTok{ b}\OperatorTok{[}\DecValTok{11}\OperatorTok{],}\NormalTok{ ans }\OperatorTok{=} \DecValTok{0}\OperatorTok{;}

\DataTypeTok{void}\NormalTok{ check}\OperatorTok{()} \OperatorTok{\{}
    \DataTypeTok{int}\NormalTok{ rem }\OperatorTok{=} \DecValTok{1}\OperatorTok{;}
    \ControlFlowTok{for} \OperatorTok{(}\DataTypeTok{int}\NormalTok{ i }\OperatorTok{=} \DecValTok{0}\OperatorTok{;}\NormalTok{ i }\OperatorTok{\textless{}}\NormalTok{ n}\OperatorTok{;} \OperatorTok{++}\NormalTok{i}\OperatorTok{)}
\NormalTok{        rem }\OperatorTok{=} \OperatorTok{(}\NormalTok{rem }\OperatorTok{*}\NormalTok{ b}\OperatorTok{[}\NormalTok{i}\OperatorTok{])} \OperatorTok{\%} \DecValTok{2}\OperatorTok{;}
\NormalTok{    ans }\OperatorTok{+=}\NormalTok{ rem }\OperatorTok{==} \DecValTok{0}\OperatorTok{;}
\OperatorTok{\}}

\DataTypeTok{void}\NormalTok{ backtrack}\OperatorTok{(}\DataTypeTok{int}\NormalTok{ i}\OperatorTok{)} \OperatorTok{\{}
    \ControlFlowTok{if} \OperatorTok{(}\NormalTok{i }\OperatorTok{==}\NormalTok{ n}\OperatorTok{)} \OperatorTok{\{}
\NormalTok{        check}\OperatorTok{();}
        \ControlFlowTok{return}\OperatorTok{;}
    \OperatorTok{\}}
    \CommentTok{// Duyệt các khả năng của b[i].}
    \ControlFlowTok{for} \OperatorTok{(}\DataTypeTok{int}\NormalTok{ j }\OperatorTok{=}\NormalTok{ a}\OperatorTok{[}\NormalTok{i}\OperatorTok{]} \OperatorTok{{-}} \DecValTok{1}\OperatorTok{;}\NormalTok{ j }\OperatorTok{\textless{}=}\NormalTok{ a}\OperatorTok{[}\NormalTok{i}\OperatorTok{]} \OperatorTok{+} \DecValTok{1}\OperatorTok{;} \OperatorTok{++}\NormalTok{j}\OperatorTok{)} \OperatorTok{\{}
\NormalTok{        b}\OperatorTok{[}\NormalTok{i}\OperatorTok{]} \OperatorTok{=}\NormalTok{ j}\OperatorTok{;}
\NormalTok{        backtrack}\OperatorTok{(}\NormalTok{i }\OperatorTok{+} \DecValTok{1}\OperatorTok{);}
    \OperatorTok{\}}
\OperatorTok{\}}
\end{Highlighting}
\end{Shaded}

\subsection{Bài 10: Tìm dãy con liên tiếp dài nhất có đúng K số nguyên
tố phân
biệt}\label{buxe0i-10-tuxecm-duxe3y-con-liuxean-tiux1ebfp-duxe0i-nhux1ea5t-cuxf3-ux111uxfang-k-sux1ed1-nguyuxean-tux1ed1-phuxe2n-biux1ec7t}

Sử dụng 2 con trỏ để duyệt dãy.

\subsubsection{Hai con trỏ là gì?}\label{hai-con-trux1ecf-luxe0-guxec}

\begin{verbatim}
a1 a2 a3 a4 a5 a6 a7 a8 a9 a10
^------------^
|            |
left         right
\end{verbatim}

Khi dịch sang phải, ta thấy rằng:

\begin{verbatim}
a1 a2 a3 a4 a5 a6 a7 a8 a9 a10
   ^------------^
\end{verbatim}

So với trường hợp trước, ta thấy rằng: - Mất đi a1 và thêm vào a6. - Chỉ
cần kiểm tra lại a1 và a6 \(\Rightarrow  O(1)\), (không cần duyệt lại
toàn bộ dãy).

Quay lại bài toán:

B1: Đánh dấu các vị trí không phải số nguyên tố trong mảng là 0.

Cửa sổ trượt cần quản lý một đoạn dài nhất có đúng K số nguyên tố phân
biệt. - Khi dịch sang phải một ô mới, thì có hai trường hợp: + Nếu
\(a[right + 1]\) là số nguyên tố mới thì số lượng số nguyên tố phân biệt
tăng lên. + Nếu số lượng số nguyên tố phân biệt vượt quá K thì cần cắt
bỏ phần đầu dãy để giảm số lượng số nguyên tố phân biệt về \(K\). + Nếu
\(a[right + 1]\) là số nguyên tố cũ hoặc không phải số nguyên tố thì
không cần làm gì cả. \(\Rightarrow\) Cần mảng thống kê số lần xuất hiện
của các số nguyên tố trong cửa sổ trượt. - Nếu số lượng số nguyên tố
phân biệt bằng K thì cập nhật kết quả.

\begin{Shaded}
\begin{Highlighting}[]
\DataTypeTok{void}\NormalTok{ solve}\OperatorTok{()} \OperatorTok{\{}
    \DataTypeTok{int}\NormalTok{ n}\OperatorTok{,}\NormalTok{ k}\OperatorTok{;}
\NormalTok{    cin }\OperatorTok{\textgreater{}\textgreater{}}\NormalTok{ n }\OperatorTok{\textgreater{}\textgreater{}}\NormalTok{ k}\OperatorTok{;}
\NormalTok{    vector}\OperatorTok{\textless{}}\DataTypeTok{int}\OperatorTok{\textgreater{}}\NormalTok{ a}\OperatorTok{(}\NormalTok{n}\OperatorTok{);}
    \ControlFlowTok{for} \OperatorTok{(}\DataTypeTok{int} \OperatorTok{\&}\NormalTok{x }\OperatorTok{:}\NormalTok{ a}\OperatorTok{)}\NormalTok{ cin }\OperatorTok{\textgreater{}\textgreater{}}\NormalTok{ x}\OperatorTok{;}

\NormalTok{    vector}\OperatorTok{\textless{}}\DataTypeTok{int}\OperatorTok{\textgreater{}}\NormalTok{ is\_prime}\OperatorTok{(}\DecValTok{1000001}\OperatorTok{,} \DecValTok{1}\OperatorTok{);}
\NormalTok{    is\_prime}\OperatorTok{[}\DecValTok{0}\OperatorTok{]} \OperatorTok{=}\NormalTok{ is\_prime}\OperatorTok{[}\DecValTok{1}\OperatorTok{]} \OperatorTok{=} \DecValTok{0}\OperatorTok{;}
    \ControlFlowTok{for} \OperatorTok{(}\DataTypeTok{int}\NormalTok{ i }\OperatorTok{=} \DecValTok{2}\OperatorTok{;}\NormalTok{ i }\OperatorTok{\textless{}=} \DecValTok{1000000}\OperatorTok{;} \OperatorTok{++}\NormalTok{i}\OperatorTok{)} \OperatorTok{\{}
        \ControlFlowTok{if} \OperatorTok{(}\NormalTok{is\_prime}\OperatorTok{[}\NormalTok{i}\OperatorTok{])} \OperatorTok{\{}
            \ControlFlowTok{for} \OperatorTok{(}\DataTypeTok{int}\NormalTok{ j }\OperatorTok{=}\NormalTok{ i }\OperatorTok{*} \DecValTok{2}\OperatorTok{;}\NormalTok{ j }\OperatorTok{\textless{}=} \DecValTok{1000000}\OperatorTok{;}\NormalTok{ j }\OperatorTok{+=}\NormalTok{ i}\OperatorTok{)}
\NormalTok{                is\_prime}\OperatorTok{[}\NormalTok{j}\OperatorTok{]} \OperatorTok{=} \DecValTok{0}\OperatorTok{;}
        \OperatorTok{\}}
    \OperatorTok{\}}

    \CommentTok{// B1: Đánh dấu các vị trí không phải số nguyên tố trong mảng là 0.}
    \ControlFlowTok{for} \OperatorTok{(}\DataTypeTok{int} \OperatorTok{\&}\NormalTok{x}\OperatorTok{:}\NormalTok{ a}\OperatorTok{)}
        \ControlFlowTok{if} \OperatorTok{(!}\NormalTok{is\_prime}\OperatorTok{[}\NormalTok{x}\OperatorTok{])}\NormalTok{ x }\OperatorTok{=} \DecValTok{0}\OperatorTok{;}
    
    \CommentTok{// B2: Cửa sổ trượt.}
    \DataTypeTok{int}\NormalTok{ ans }\OperatorTok{=} \OperatorTok{{-}}\DecValTok{1}\OperatorTok{,}\NormalTok{ L }\OperatorTok{=} \DecValTok{0}\OperatorTok{,}\NormalTok{ R }\OperatorTok{=} \OperatorTok{{-}}\DecValTok{1}\OperatorTok{;}
\NormalTok{    vector}\OperatorTok{\textless{}}\DataTypeTok{int}\OperatorTok{\textgreater{}}\NormalTok{ cnt}\OperatorTok{(}\DecValTok{1000001}\OperatorTok{,} \DecValTok{0}\OperatorTok{);} \CommentTok{// Thống kê số lần xuất hiện của các số nguyên tố trong cửa sổ trượt.}
    \DataTypeTok{int}\NormalTok{ primes }\OperatorTok{=} \DecValTok{0}\OperatorTok{;} \CommentTok{// Số lượng số nguyên tố phân biệt trong cửa sổ trượt.}
    \ControlFlowTok{while} \OperatorTok{(}\DecValTok{1}\OperatorTok{)} \OperatorTok{\{}
        \CommentTok{// Dịch sang phải một ô.}
\NormalTok{        R}\OperatorTok{++;}
        \ControlFlowTok{if} \OperatorTok{(}\NormalTok{R }\OperatorTok{==}\NormalTok{ n}\OperatorTok{)} \ControlFlowTok{break}\OperatorTok{;}
        \ControlFlowTok{if} \OperatorTok{(}\NormalTok{a}\OperatorTok{[}\NormalTok{R}\OperatorTok{]} \OperatorTok{!=} \DecValTok{0} \KeywordTok{and}\NormalTok{ cnt}\OperatorTok{[}\NormalTok{a}\OperatorTok{[}\NormalTok{R}\OperatorTok{]]} \OperatorTok{==} \DecValTok{0}\OperatorTok{)}\NormalTok{ primes}\OperatorTok{++;}
\NormalTok{        cnt}\OperatorTok{[}\NormalTok{a}\OperatorTok{[}\NormalTok{R}\OperatorTok{]]++;}

        \CommentTok{// Nếu số lượng số nguyên tố phân biệt vượt quá K thì cần cắt bỏ phần đầu dãy để giảm số lượng số nguyên tố phân biệt về K.}
        \ControlFlowTok{while} \OperatorTok{(}\NormalTok{primes }\OperatorTok{\textgreater{}}\NormalTok{ k}\OperatorTok{)} \OperatorTok{\{}
            \ControlFlowTok{if} \OperatorTok{(}\NormalTok{a}\OperatorTok{[}\NormalTok{L}\OperatorTok{]} \OperatorTok{!=} \DecValTok{0}\OperatorTok{)} \OperatorTok{\{}
\NormalTok{                cnt}\OperatorTok{[}\NormalTok{a}\OperatorTok{[}\NormalTok{L}\OperatorTok{]]{-}{-};}
                \ControlFlowTok{if} \OperatorTok{(}\NormalTok{cnt}\OperatorTok{[}\NormalTok{a}\OperatorTok{[}\NormalTok{L}\OperatorTok{]]} \OperatorTok{==} \DecValTok{0}\OperatorTok{)}\NormalTok{ primes}\OperatorTok{{-}{-};}
            \OperatorTok{\}}
\NormalTok{            L}\OperatorTok{++;}
        \OperatorTok{\}}

        \CommentTok{// Phải đảm bảo đoạn [L, R] không vi phạm điều kiện.}

        \CommentTok{// Nếu số lượng số nguyên tố phân biệt bằng K thì cập nhật kết quả.}
        \ControlFlowTok{if} \OperatorTok{(}\NormalTok{primes }\OperatorTok{==}\NormalTok{ k}\OperatorTok{)}\NormalTok{ ans }\OperatorTok{=}\NormalTok{ max}\OperatorTok{(}\NormalTok{ans}\OperatorTok{,}\NormalTok{ R }\OperatorTok{{-}}\NormalTok{ L }\OperatorTok{+} \DecValTok{1}\OperatorTok{);}
    \OperatorTok{\}}
\NormalTok{    cout }\OperatorTok{\textless{}\textless{}}\NormalTok{ ans }\OperatorTok{\textless{}\textless{}} \CharTok{\textquotesingle{}}\SpecialCharTok{\textbackslash{}n}\CharTok{\textquotesingle{}}\OperatorTok{;}
\OperatorTok{\}}
\end{Highlighting}
\end{Shaded}

\subsection{Bài 11: Số lượng số hạn
chế}\label{buxe0i-11-sux1ed1-lux1b0ux1ee3ng-sux1ed1-hux1ea1n-chux1ebf}

\begin{itemize}
\tightlist
\item
  Só hạn chế là số có đúng 4 ước trong đó có 2 ước nguyên tố.
\item
  Đếm trong đoạn \([L, R]\) có bao nhiêu số hạn chế. \(\Leftrightarrow\)
  Quy về đếm số hạn chế trong đoạn \([1, R]\) - đếm số hạn chế trong
  đoạn \([1, L - 1]\).
\item
  Gọi \(cnt[x]\) là số lượng số hạn chế trong đoạn {[}1, x{]}.
\item
  \(cnt[x] = cnt[x - 1] + (\text{x có đúng 4 ước trong đó có 2 ước nguyên tố}).\)
\end{itemize}

\subsubsection{Cách kiểm tra x có đúng 4 ước trong đó có 2 ước nguyên
tố:}\label{cuxe1ch-kiux1ec3m-tra-x-cuxf3-ux111uxfang-4-ux1b0ux1edbc-trong-ux111uxf3-cuxf3-2-ux1b0ux1edbc-nguyuxean-tux1ed1}

x có 4 ước thì phải thuộc một trong hai dạng: -
\(x = p^3 => 1, p, p^2, x\) - \(x = p . q => 1, p, q, x\)

\(x\) phải là tích 2 số nguyên tố phân biết, tức là \(x = p * q\) với
\(p, q\) là số nguyên tố phân biệt. =\textgreater{} Theo công thức đém
ước, số ước của \(x = (a + 1) * (b + 1)\) với \(a, b\) là số mũ của
\(p, q\).

\begin{itemize}
\tightlist
\item
  Cách 1: Sàng ước =\textgreater{} Tính được mảng số ước của các số từ
  \(1\) đến \(10^6\) =\textgreater{} \(uoc[x]\) = số ước của x.

  \begin{itemize}
  \tightlist
  \item
    Nếu \(uoc[x] == 4\) và \(x\) không phải số lập phương thì \(x\) là
    số hạn chế.
  \end{itemize}
\item
  Cách 2: Tự nghĩ.
\end{itemize}

\begin{Shaded}
\begin{Highlighting}[]
\DataTypeTok{int}\NormalTok{ uoc}\OperatorTok{[}\DecValTok{200005}\OperatorTok{],}\NormalTok{ cnt}\OperatorTok{[}\DecValTok{200005}\OperatorTok{];}

\DataTypeTok{bool}\NormalTok{ hanche}\OperatorTok{(}\DataTypeTok{int}\NormalTok{ x}\OperatorTok{)} \OperatorTok{\{}
    \ControlFlowTok{if} \OperatorTok{(}\NormalTok{uoc}\OperatorTok{[}\NormalTok{x}\OperatorTok{]} \OperatorTok{!=} \DecValTok{4}\OperatorTok{)} \ControlFlowTok{return} \KeywordTok{false}\OperatorTok{;}
    \DataTypeTok{int}\NormalTok{ k }\OperatorTok{=}\NormalTok{ cbrtl}\OperatorTok{(}\NormalTok{x}\OperatorTok{);}
    \ControlFlowTok{return}\NormalTok{ k }\OperatorTok{*}\NormalTok{ k }\OperatorTok{*}\NormalTok{ k }\OperatorTok{!=}\NormalTok{ x}\OperatorTok{;}
\OperatorTok{\}}

\DataTypeTok{void}\NormalTok{ solve}\OperatorTok{()} \OperatorTok{\{}
    \CommentTok{// Sàng nguyên tố =\textgreater{} O(n * log(log(n))). \textasciitilde{} 10\^{}7.}
    \CommentTok{// Sàng ước =\textgreater{} Rất quan trọng =\textgreater{} O(n * log(n)). \textasciitilde{} 10\^{}6.}
    \ControlFlowTok{for} \OperatorTok{(}\DataTypeTok{int}\NormalTok{ i }\OperatorTok{=} \DecValTok{1}\OperatorTok{;}\NormalTok{ i }\OperatorTok{\textless{}=} \DecValTok{200000}\OperatorTok{;} \OperatorTok{++}\NormalTok{i}\OperatorTok{)}
        \ControlFlowTok{for} \OperatorTok{(}\DataTypeTok{int}\NormalTok{ j }\OperatorTok{=}\NormalTok{ i}\OperatorTok{;}\NormalTok{ j }\OperatorTok{\textless{}=} \DecValTok{200000}\OperatorTok{;}\NormalTok{ j }\OperatorTok{+=}\NormalTok{ i}\OperatorTok{)}
\NormalTok{            uoc}\OperatorTok{[}\NormalTok{j}\OperatorTok{]++;}
    \CommentTok{// Tính cnt.}
    \ControlFlowTok{for} \OperatorTok{(}\DataTypeTok{int}\NormalTok{ i }\OperatorTok{=} \DecValTok{1}\OperatorTok{;}\NormalTok{ i }\OperatorTok{\textless{}=} \DecValTok{200000}\OperatorTok{;} \OperatorTok{++}\NormalTok{i}\OperatorTok{)}
\NormalTok{        cnt}\OperatorTok{[}\NormalTok{i}\OperatorTok{]} \OperatorTok{=}\NormalTok{ cnt}\OperatorTok{[}\NormalTok{i }\OperatorTok{{-}} \DecValTok{1}\OperatorTok{]} \OperatorTok{+}\NormalTok{ hanche}\OperatorTok{(}\NormalTok{i}\OperatorTok{);}
    \DataTypeTok{int}\NormalTok{ t}\OperatorTok{;}
\NormalTok{    cin }\OperatorTok{\textgreater{}\textgreater{}}\NormalTok{ t}\OperatorTok{;}
    \ControlFlowTok{while} \OperatorTok{(}\NormalTok{t}\OperatorTok{{-}{-})} \OperatorTok{\{}
        \DataTypeTok{int}\NormalTok{ l}\OperatorTok{,}\NormalTok{ r}\OperatorTok{;}
\NormalTok{        cin }\OperatorTok{\textgreater{}\textgreater{}}\NormalTok{ l }\OperatorTok{\textgreater{}\textgreater{}}\NormalTok{ r}\OperatorTok{;}
\NormalTok{        cout }\OperatorTok{\textless{}\textless{}}\NormalTok{ cnt}\OperatorTok{[}\NormalTok{r}\OperatorTok{]} \OperatorTok{{-}}\NormalTok{ cnt}\OperatorTok{[}\NormalTok{l }\OperatorTok{{-}} \DecValTok{1}\OperatorTok{]} \OperatorTok{\textless{}\textless{}} \CharTok{\textquotesingle{}}\SpecialCharTok{\textbackslash{}n}\CharTok{\textquotesingle{}}\OperatorTok{;}
    \OperatorTok{\}}
\OperatorTok{\}}
\end{Highlighting}
\end{Shaded}

\subsection{Bài 12: Nguyên tố tương
đương}\label{buxe0i-12-nguyuxean-tux1ed1-tux1b0ux1a1ng-ux111ux1b0ux1a1ng}

\(A, B\) là nguyên tố tương đương khi \(A\) và \(B\) cùng tập ước nguyên
tố \(\Leftrightarrow\) Tích các ước nguyên tố của \(A\) = Tích các ước
nguyên tố của \(B\).

\(\Rightarrow\) Sử dụng sàng nguyên tố + sàng ước để tính tích cách ước
nguyên tố của các số từ 1 đến \(10^6\).

Quy về bài toán đếm số cặp \((A < B)\) có \(product[A] == product[B]\)
\(\Rightarrow\) Bài toán thống kê cơ bản.

\begin{Shaded}
\begin{Highlighting}[]
\DataTypeTok{int}\NormalTok{ product}\OperatorTok{[}\DecValTok{1000005}\OperatorTok{],}\NormalTok{ cnt}\OperatorTok{[}\DecValTok{1000005}\OperatorTok{];}
\DataTypeTok{bool}\NormalTok{ is\_prime}\OperatorTok{[}\DecValTok{1000005}\OperatorTok{];}

\DataTypeTok{void}\NormalTok{ solve}\OperatorTok{()} \OperatorTok{\{}
    \CommentTok{// Sàng nguyên tố.}
\NormalTok{    fill}\OperatorTok{(}\NormalTok{is\_prime}\OperatorTok{,}\NormalTok{ is\_prime }\OperatorTok{+} \DecValTok{1000001}\OperatorTok{,} \DecValTok{1}\OperatorTok{);}
\NormalTok{    is\_prime}\OperatorTok{[}\DecValTok{0}\OperatorTok{]} \OperatorTok{=}\NormalTok{ is\_prime}\OperatorTok{[}\DecValTok{1}\OperatorTok{]} \OperatorTok{=} \DecValTok{0}\OperatorTok{;}
    \ControlFlowTok{for} \OperatorTok{(}\DataTypeTok{int}\NormalTok{ i }\OperatorTok{=} \DecValTok{2}\OperatorTok{;}\NormalTok{ i }\OperatorTok{\textless{}=} \DecValTok{1000}\OperatorTok{;} \OperatorTok{++}\NormalTok{i}\OperatorTok{)} \OperatorTok{\{}
        \ControlFlowTok{if} \OperatorTok{(}\NormalTok{is\_prime}\OperatorTok{[}\NormalTok{i}\OperatorTok{])} \OperatorTok{\{}
            \ControlFlowTok{for} \OperatorTok{(}\DataTypeTok{int}\NormalTok{ j }\OperatorTok{=}\NormalTok{ i }\OperatorTok{*}\NormalTok{ i}\OperatorTok{;}\NormalTok{ j }\OperatorTok{\textless{}=} \DecValTok{1000000}\OperatorTok{;}\NormalTok{ j }\OperatorTok{+=}\NormalTok{ i}\OperatorTok{)}
\NormalTok{                is\_prime}\OperatorTok{[}\NormalTok{j}\OperatorTok{]} \OperatorTok{=} \DecValTok{0}\OperatorTok{;}
        \OperatorTok{\}}
    \OperatorTok{\}}
    \CommentTok{// Sàng ước =\textgreater{} tính product.}
\NormalTok{    fill}\OperatorTok{(}\NormalTok{product}\OperatorTok{,}\NormalTok{ product }\OperatorTok{+} \DecValTok{1000001}\OperatorTok{,} \DecValTok{1}\OperatorTok{);}
    \ControlFlowTok{for} \OperatorTok{(}\DataTypeTok{int}\NormalTok{ i }\OperatorTok{=} \DecValTok{1}\OperatorTok{;}\NormalTok{ i }\OperatorTok{\textless{}=} \DecValTok{1000000}\OperatorTok{;} \OperatorTok{++}\NormalTok{i}\OperatorTok{)}
        \ControlFlowTok{if} \OperatorTok{(}\NormalTok{is\_prime}\OperatorTok{[}\NormalTok{i}\OperatorTok{])}
            \ControlFlowTok{for} \OperatorTok{(}\DataTypeTok{int}\NormalTok{ j }\OperatorTok{=}\NormalTok{ i}\OperatorTok{;}\NormalTok{ j }\OperatorTok{\textless{}=} \DecValTok{1000000}\OperatorTok{;}\NormalTok{ j }\OperatorTok{+=}\NormalTok{ i}\OperatorTok{)}
\NormalTok{                product}\OperatorTok{[}\NormalTok{j}\OperatorTok{]} \OperatorTok{*=}\NormalTok{ i}\OperatorTok{;}
    \DataTypeTok{int}\NormalTok{ t}\OperatorTok{;}
\NormalTok{    cin }\OperatorTok{\textgreater{}\textgreater{}}\NormalTok{ t}\OperatorTok{;}
    \ControlFlowTok{while} \OperatorTok{(}\NormalTok{t}\OperatorTok{{-}{-})} \OperatorTok{\{}
        \DataTypeTok{int}\NormalTok{ a}\OperatorTok{,}\NormalTok{ b}\OperatorTok{;}
\NormalTok{        cin }\OperatorTok{\textgreater{}\textgreater{}}\NormalTok{ a }\OperatorTok{\textgreater{}\textgreater{}}\NormalTok{ b}\OperatorTok{;}
\NormalTok{        memset}\OperatorTok{(}\NormalTok{cnt}\OperatorTok{,} \DecValTok{0}\OperatorTok{,} \KeywordTok{sizeof}\OperatorTok{(}\NormalTok{cnt}\OperatorTok{));}
        \DataTypeTok{long} \DataTypeTok{long}\NormalTok{ ans }\OperatorTok{=} \DecValTok{0}\OperatorTok{;}
        \ControlFlowTok{for} \OperatorTok{(}\DataTypeTok{int}\NormalTok{ i }\OperatorTok{=}\NormalTok{ a}\OperatorTok{;}\NormalTok{ i }\OperatorTok{\textless{}=}\NormalTok{ b}\OperatorTok{;} \OperatorTok{++}\NormalTok{i}\OperatorTok{)} \OperatorTok{\{}
\NormalTok{            ans }\OperatorTok{+=}\NormalTok{ cnt}\OperatorTok{[}\NormalTok{product}\OperatorTok{[}\NormalTok{i}\OperatorTok{]];}
\NormalTok{            cnt}\OperatorTok{[}\NormalTok{product}\OperatorTok{[}\NormalTok{i}\OperatorTok{]]++;}
        \OperatorTok{\}}
\NormalTok{        cout }\OperatorTok{\textless{}\textless{}}\NormalTok{ ans }\OperatorTok{\textless{}\textless{}} \CharTok{\textquotesingle{}}\SpecialCharTok{\textbackslash{}n}\CharTok{\textquotesingle{}}\OperatorTok{;}
    \OperatorTok{\}}
\OperatorTok{\}}
\end{Highlighting}
\end{Shaded}

\subsection{Bài 15: Dãy tương tự
hard}\label{buxe0i-15-duxe3y-tux1b0ux1a1ng-tux1ef1-hard}

\begin{itemize}
\tightlist
\item
  Dãy tương tự khi \(|b[i] - a[i]| <= 1\) với mọi \(i\).
\item
  \(b[i]\) có thể nhận một trong 3 giá trị:
  \(a[i] - 1, a[i], a[i] + 1\).
\item
  Chỉ cần có một giá trị \(b[i]\) chẵn thì tích các số trong dãy b là
  chẵn. \(\Rightarrow\) Nếu không có giá trị \(b[i]\) chẵn thì tích các
  số trong dãy b là lẻ.
\item
  Đếm bằng cách bù trừ \(\Rightarrow\) Lấy tổng số dãy tương tự - số dãy
  tương tự không có giá trị chẵn.
\item
  Tổng số dãy tương tự = \(3^n\).
\item
  Số dãy tương tự không có giá trị chẵn là: \(2^k\) với \(k\) là số
  lượng vị trí \(a[i]\) là chẵn. (Tự chứng minh)
\item
  Đáp án là \(3^n - 2^k\).
\end{itemize}
