% Options for packages loaded elsewhere
\PassOptionsToPackage{unicode}{hyperref}
\PassOptionsToPackage{hyphens}{url}
%
\documentclass[]{article}

\input{config.tex}

\author{}
\date{}

\begin{document}

\section{Bài 1:}\label{buxe0i-1}

Tìm số hoàn hảo thứ \(K\)

\subsection{Ý tưởng tìm kiếm nhị
phân}\label{uxfd-tux1b0ux1edfng-tuxecm-kiux1ebfm-nhux1ecb-phuxe2n}

Tìm kiếm nhị phân từ \(1\) đến \(MAX\).

Số hoàn hảo thứ \(K\) là số tự nhiên nhỏ nhất có ít nhất \(K\) số hoàn
hảo bé hơn hoặc bằng nó.

Ví dụ: \(37\) là số thứ \(3\) vì nó có \(3\) số hoàn hảo bé hơn hoặc
bắng nó là \(19\), \(28\), \(37\).

Khi đó chúng ta có thể sử dụng DP Digit để đếm số lượng số hoàn hảo bé
hơn hoặc bằng \(N\).

\begin{Shaded}
\begin{Highlighting}[]
\NormalTok{ll dp}\OperatorTok{[}\NormalTok{i}\OperatorTok{][}\NormalTok{sum}\OperatorTok{][}\NormalTok{ok}\OperatorTok{];}
\end{Highlighting}
\end{Shaded}

\begin{itemize}
\tightlist
\item
  \(i\) là vị trí hiện tại.
\item
  \(sum\) là tổng các chữ số.
\item
  \(ok\) là biến kiểm tra xem hiện tại đang bé hơn hay là bằng \(N\).
\end{itemize}

\begin{Shaded}
\begin{Highlighting}[]
\NormalTok{ll L }\OperatorTok{=} \DecValTok{1}\OperatorTok{,}\NormalTok{ R }\OperatorTok{=}\NormalTok{ MAX}\OperatorTok{;}
\ControlFlowTok{while} \OperatorTok{(}\NormalTok{L }\OperatorTok{\textless{}=}\NormalTok{ R}\OperatorTok{)} \OperatorTok{\{}
\NormalTok{    ll mid }\OperatorTok{=} \OperatorTok{(}\NormalTok{L }\OperatorTok{+}\NormalTok{ R}\OperatorTok{)} \OperatorTok{/} \DecValTok{2}\OperatorTok{;}
    \ControlFlowTok{if} \OperatorTok{(}\NormalTok{cal}\OperatorTok{(}\NormalTok{mid}\OperatorTok{)} \OperatorTok{\textgreater{}=}\NormalTok{ K}\OperatorTok{)} \OperatorTok{\{}
\NormalTok{        ans }\OperatorTok{=}\NormalTok{ mid}\OperatorTok{;}
\NormalTok{        R }\OperatorTok{=}\NormalTok{ mid }\OperatorTok{{-}} \DecValTok{1}\OperatorTok{;}
    \OperatorTok{\}} \ControlFlowTok{else}\NormalTok{ L }\OperatorTok{=}\NormalTok{ mid }\OperatorTok{+} \DecValTok{1}\OperatorTok{;}
\OperatorTok{\}}
\end{Highlighting}
\end{Shaded}

\subsubsection{Hạn chế}\label{hux1ea1n-chux1ebf}

MAX bị giới hạn bởi long long. Do đó, nếu kết quả vượt quá long long thì
tìm kiếm nhị phân sẽ khó khăn trong việc tính L, R, mid.

\subsection{Ý tưởng DP cấu
hình}\label{uxfd-tux1b0ux1edfng-dp-cux1ea5u-huxecnh}

Cấu hình cần tạo có dạng \(a_1, a_2, ..., a_{300}\) với
\(a_i \in \{0, 9\}\) và \(a_1+a_2+...+a_{300} = 10\)

Nếu \(a_1 = 1\) thì nó lớn hơn mọi cấu hình có
\(a_1 = 0 \to dp[2][10]\).

Nếu \(a_1 = 2\) thì nó lớn hơn mọi cấu hình có
\(a_1 = \{0, 1\} \to dp[2][10] + dp[2][9]\).

Nếu \(a_1 = 3\) thì nó lớn hơn mọi cấu hình có
\(a_1 = \{0, 1, 2\} \to dp[2][10] + dp[2][9] + dp[2][8]\).

\ldots{}

Nếu \(a_1 = 9\) thì nó lớn hơn mọi cấu hình có
\(a_1 = \{0, 1, 2, ..., 8\} \to dp[2][10] + dp[2][9] + dp[2][8] + ... + dp[2][1]\).

\begin{Shaded}
\begin{Highlighting}[]
\NormalTok{ll dp}\OperatorTok{[}\NormalTok{i}\OperatorTok{][}\NormalTok{sum}\OperatorTok{];}

\DataTypeTok{int}\NormalTok{ a}\OperatorTok{[}\DecValTok{300}\OperatorTok{];}
\NormalTok{ll K}\OperatorTok{;}

\ControlFlowTok{for} \OperatorTok{(}\DataTypeTok{int}\NormalTok{ i }\OperatorTok{=} \DecValTok{0}\OperatorTok{,}\NormalTok{ sum }\OperatorTok{=} \DecValTok{10}\OperatorTok{;}\NormalTok{ i }\OperatorTok{\textless{}} \DecValTok{300}\OperatorTok{;} \OperatorTok{++}\NormalTok{i}\OperatorTok{)}
    \ControlFlowTok{for} \OperatorTok{(}\DataTypeTok{int}\NormalTok{ j }\OperatorTok{=} \DecValTok{0}\OperatorTok{;} \OperatorTok{;} \OperatorTok{++}\NormalTok{j}\OperatorTok{)} \OperatorTok{\{}
        \ControlFlowTok{if} \OperatorTok{(}\NormalTok{K }\OperatorTok{\textgreater{}}\NormalTok{ cal}\OperatorTok{(}\NormalTok{i }\OperatorTok{+} \DecValTok{1}\OperatorTok{,}\NormalTok{ sum }\OperatorTok{{-}}\NormalTok{ j}\OperatorTok{))}
\NormalTok{            K }\OperatorTok{{-}=}\NormalTok{ cal}\OperatorTok{(}\NormalTok{i }\OperatorTok{+} \DecValTok{1}\OperatorTok{,}\NormalTok{ sum }\OperatorTok{{-}}\NormalTok{ j}\OperatorTok{);}
        \ControlFlowTok{else} \OperatorTok{\{}
\NormalTok{            a}\OperatorTok{[}\NormalTok{i}\OperatorTok{]} \OperatorTok{=}\NormalTok{ j}\OperatorTok{;}
\NormalTok{            sum }\OperatorTok{{-}=}\NormalTok{ j}\OperatorTok{;}
            \ControlFlowTok{break}\OperatorTok{;}
        \OperatorTok{\}}
    \OperatorTok{\}}
\end{Highlighting}
\end{Shaded}

\subsubsection{Cách tính
dp{[}i{]}{[}sum{]}}\label{cuxe1ch-tuxednh-dpisum}

\begin{Shaded}
\begin{Highlighting}[]
\NormalTok{ll cal}\OperatorTok{(}\DataTypeTok{int}\NormalTok{ i}\OperatorTok{,} \DataTypeTok{int}\NormalTok{ sum}\OperatorTok{)} \OperatorTok{\{}
    \ControlFlowTok{if} \OperatorTok{(}\NormalTok{i }\OperatorTok{==} \DecValTok{300}\OperatorTok{)} \ControlFlowTok{return}\NormalTok{ sum }\OperatorTok{==} \DecValTok{0}\OperatorTok{;}
    \ControlFlowTok{if} \OperatorTok{(}\NormalTok{dp}\OperatorTok{[}\NormalTok{i}\OperatorTok{][}\NormalTok{sum}\OperatorTok{]} \OperatorTok{!=} \OperatorTok{{-}}\DecValTok{1}\OperatorTok{)} \ControlFlowTok{return}\NormalTok{ dp}\OperatorTok{[}\NormalTok{i}\OperatorTok{][}\NormalTok{sum}\OperatorTok{];}
\NormalTok{    ll res }\OperatorTok{=} \DecValTok{0}\OperatorTok{;}
    \ControlFlowTok{for} \OperatorTok{(}\DataTypeTok{int}\NormalTok{ j }\OperatorTok{=} \DecValTok{0}\OperatorTok{;}\NormalTok{ j }\OperatorTok{\textless{}=}\NormalTok{ min}\OperatorTok{(}\DecValTok{9}\OperatorTok{,}\NormalTok{ sum}\OperatorTok{);} \OperatorTok{++}\NormalTok{j}\OperatorTok{)} \OperatorTok{\{}
\NormalTok{        res }\OperatorTok{+=}\NormalTok{ cal}\OperatorTok{(}\NormalTok{i }\OperatorTok{+} \DecValTok{1}\OperatorTok{,}\NormalTok{ sum }\OperatorTok{{-}}\NormalTok{ j}\OperatorTok{);}
        \ControlFlowTok{if} \OperatorTok{(}\NormalTok{res }\OperatorTok{\textgreater{}} \FloatTok{1e18}\OperatorTok{)}\NormalTok{ res }\OperatorTok{=} \FloatTok{1e18}\OperatorTok{;}
    \OperatorTok{\}}
    \ControlFlowTok{return}\NormalTok{ dp}\OperatorTok{[}\NormalTok{i}\OperatorTok{][}\NormalTok{sum}\OperatorTok{]} \OperatorTok{=}\NormalTok{ res}\OperatorTok{;}
\OperatorTok{\}}
\end{Highlighting}
\end{Shaded}

\section{Bài 3:}\label{buxe0i-3}

Tìm \(A + B + C = N\) thoả mãn \(A\) chia hết cho \(a\), \(B\) chia hết
cho \(b\), \(C\) chia hết cho \(c\).

\subsection{Ý tưởng}\label{uxfd-tux1b0ux1edfng}

Nhắc lại phép cộng tiểu học:

\begin{verbatim}
   123
+  456
+  789
------
  1368
\end{verbatim}

Nếu ta đặt \(A = a_1a_2...a_9\), \(B = b_1b_2...b_9\),
\(C = c_1c_2...c_9\) thì ta có:

\begin{verbatim}
   a1a2...a9
+  b1b2...b9
+  c1c2...c9
---------------
   n1n2...n9
\end{verbatim}

Vì phép cộng sẽ cộng từ hàng đơn vị trở lên, cho nên chúng ta cũng phải
xây dựng từ hàng đơn vị trở lên.

Mỗi hàng, chúng ta sẽ thử cả \(a_i, b_i, c_i\) từ \(0\) đến \(9\) sao
cho \(a_i + b_i + c_i + nho = n_i\) với \(nho\) là số nhớ từ hàng trước.

\begin{Shaded}
\begin{Highlighting}[]
\NormalTok{string n}\OperatorTok{;}
\NormalTok{ll dp}\OperatorTok{[}\DecValTok{10}\OperatorTok{][}\DecValTok{3}\OperatorTok{];}

\NormalTok{ll cal}\OperatorTok{(}\DataTypeTok{int}\NormalTok{ i}\OperatorTok{,} \DataTypeTok{int}\NormalTok{ carry}\OperatorTok{)} \OperatorTok{\{}
    \ControlFlowTok{if} \OperatorTok{(}\NormalTok{i }\OperatorTok{==} \DecValTok{0}\OperatorTok{)} \ControlFlowTok{return}\NormalTok{ carry }\OperatorTok{==} \DecValTok{0}\OperatorTok{;}
    \ControlFlowTok{if} \OperatorTok{(}\NormalTok{dp}\OperatorTok{[}\NormalTok{i}\OperatorTok{][}\NormalTok{carry}\OperatorTok{]} \OperatorTok{!=} \OperatorTok{{-}}\DecValTok{1}\OperatorTok{)} \ControlFlowTok{return}\NormalTok{ dp}\OperatorTok{[}\NormalTok{i}\OperatorTok{][}\NormalTok{carry}\OperatorTok{];}
\NormalTok{    ll res }\OperatorTok{=} \DecValTok{0}\OperatorTok{;}
    \ControlFlowTok{for} \OperatorTok{(}\DataTypeTok{int}\NormalTok{ a }\OperatorTok{=} \DecValTok{0}\OperatorTok{;}\NormalTok{ a }\OperatorTok{\textless{}=} \DecValTok{9}\OperatorTok{;} \OperatorTok{++}\NormalTok{a}\OperatorTok{)}
        \ControlFlowTok{for} \OperatorTok{(}\DataTypeTok{int}\NormalTok{ b }\OperatorTok{=} \DecValTok{0}\OperatorTok{;}\NormalTok{ b }\OperatorTok{\textless{}=} \DecValTok{9}\OperatorTok{;} \OperatorTok{++}\NormalTok{b}\OperatorTok{)}
            \ControlFlowTok{for} \OperatorTok{(}\DataTypeTok{int}\NormalTok{ c }\OperatorTok{=} \DecValTok{0}\OperatorTok{;}\NormalTok{ c }\OperatorTok{\textless{}=} \DecValTok{9}\OperatorTok{;} \OperatorTok{++}\NormalTok{c}\OperatorTok{)} \OperatorTok{\{}
                \DataTypeTok{int}\NormalTok{ sum }\OperatorTok{=}\NormalTok{ a }\OperatorTok{+}\NormalTok{ b }\OperatorTok{+}\NormalTok{ c }\OperatorTok{+}\NormalTok{ carry}\OperatorTok{;}
                \ControlFlowTok{if} \OperatorTok{(}\NormalTok{sum }\OperatorTok{\%} \DecValTok{10} \OperatorTok{!=}\NormalTok{ n}\OperatorTok{[}\NormalTok{i }\OperatorTok{{-}} \DecValTok{1}\OperatorTok{]} \OperatorTok{{-}} \CharTok{\textquotesingle{}0\textquotesingle{}}\OperatorTok{)} \ControlFlowTok{continue}\OperatorTok{;}
\NormalTok{                res }\OperatorTok{+=}\NormalTok{ cal}\OperatorTok{(}\NormalTok{i }\OperatorTok{{-}} \DecValTok{1}\OperatorTok{,}\NormalTok{ sum }\OperatorTok{/} \DecValTok{10}\OperatorTok{);}
            \OperatorTok{\}}
    \ControlFlowTok{return}\NormalTok{ dp}\OperatorTok{[}\NormalTok{i}\OperatorTok{][}\NormalTok{carry}\OperatorTok{]} \OperatorTok{=}\NormalTok{ res}\OperatorTok{;}
\OperatorTok{\}}

\NormalTok{cal}\OperatorTok{(}\NormalTok{n}\OperatorTok{.}\NormalTok{size}\OperatorTok{(),} \DecValTok{0}\OperatorTok{)}
\end{Highlighting}
\end{Shaded}

Bài toán điển hình cho dạng này là:

\begin{verbatim}
   1?3?5??
+   ??4??8
-----------
   ??7??39
\end{verbatim}

Đếm số cách thay thế dấu \texttt{?} bằng các chữ số từ \(0\) đến \(9\)
sao cho phép cộng thỏa mãn.

Quay lại bài toán gốc:

\begin{itemize}
\tightlist
\item
  \(A\) chia hết cho \(a\)
\item
  \(B\) chia hết cho \(b\)
\item
  \(C\) chia hết cho \(c\)
\end{itemize}

Nhiều bạn sẽ nghĩ đến \(dp[i][nho][du1][du2][du3]\) với
\(du1 = A \% a\), \(du2 = B \% b\), \(du3 = C \% c\).

Độ phức tạp sẽ là
\(O(10 * 3 * 31 * 31 * 31 * 10^3 * T) \approx 8.10^8 * T\) với \(T\) là
số test.

Ăn được \(T\le 3\) test.

Với \(T \le 1000\) thì sao?

Phân tích \(N = a * x + b * y + c * z\). Khi đó ta sẽ xây dựng
\(x, y, z\) thay vì \(A, B, C\).

\begin{verbatim}
   x1x2...x9   * a
+  y1y2...y9   * b
+  z1z2...z9   * c
---------------
   n1n2...n9
\end{verbatim}

Chúng ta sẽ không cần số dư cho \(A, B, C\) nữa nhưng nhớ sẽ tăng lên.

\begin{Shaded}
\begin{Highlighting}[]
\NormalTok{string n}\OperatorTok{;}
\NormalTok{ll dp}\OperatorTok{[}\DecValTok{10}\OperatorTok{][}\DecValTok{100}\OperatorTok{];} \CommentTok{// max(carry) = 31 * 9 * 3 / 10 \textasciitilde{} 83}

\NormalTok{ll cal}\OperatorTok{(}\DataTypeTok{int}\NormalTok{ i}\OperatorTok{,} \DataTypeTok{int}\NormalTok{ carry}\OperatorTok{)} \OperatorTok{\{}
    \ControlFlowTok{if} \OperatorTok{(}\NormalTok{i }\OperatorTok{==} \DecValTok{0}\OperatorTok{)} \ControlFlowTok{return}\NormalTok{ carry }\OperatorTok{==} \DecValTok{0}\OperatorTok{;}
    \ControlFlowTok{if} \OperatorTok{(}\NormalTok{dp}\OperatorTok{[}\NormalTok{i}\OperatorTok{][}\NormalTok{carry}\OperatorTok{]} \OperatorTok{!=} \OperatorTok{{-}}\DecValTok{1}\OperatorTok{)} \ControlFlowTok{return}\NormalTok{ dp}\OperatorTok{[}\NormalTok{i}\OperatorTok{][}\NormalTok{carry}\OperatorTok{];}
\NormalTok{    ll res }\OperatorTok{=} \DecValTok{0}\OperatorTok{;}
    \ControlFlowTok{for} \OperatorTok{(}\DataTypeTok{int}\NormalTok{ x }\OperatorTok{=} \DecValTok{0}\OperatorTok{;}\NormalTok{ x }\OperatorTok{\textless{}=} \DecValTok{9}\OperatorTok{;} \OperatorTok{++}\NormalTok{x}\OperatorTok{)}
        \ControlFlowTok{for} \OperatorTok{(}\DataTypeTok{int}\NormalTok{ y }\OperatorTok{=} \DecValTok{0}\OperatorTok{;}\NormalTok{ y }\OperatorTok{\textless{}=} \DecValTok{9}\OperatorTok{;} \OperatorTok{++}\NormalTok{y}\OperatorTok{)}
            \ControlFlowTok{for} \OperatorTok{(}\DataTypeTok{int}\NormalTok{ z }\OperatorTok{=} \DecValTok{0}\OperatorTok{;}\NormalTok{ z }\OperatorTok{\textless{}=} \DecValTok{9}\OperatorTok{;} \OperatorTok{++}\NormalTok{z}\OperatorTok{)} \OperatorTok{\{}
                \DataTypeTok{int}\NormalTok{ sum }\OperatorTok{=}\NormalTok{ x }\OperatorTok{*}\NormalTok{ a }\OperatorTok{+}\NormalTok{ y }\OperatorTok{*}\NormalTok{ b }\OperatorTok{+}\NormalTok{ z }\OperatorTok{*}\NormalTok{ c }\OperatorTok{+}\NormalTok{ carry}\OperatorTok{;}
                \ControlFlowTok{if} \OperatorTok{(}\NormalTok{sum }\OperatorTok{\%} \DecValTok{10} \OperatorTok{!=}\NormalTok{ n}\OperatorTok{[}\NormalTok{i }\OperatorTok{{-}} \DecValTok{1}\OperatorTok{]} \OperatorTok{{-}} \CharTok{\textquotesingle{}0\textquotesingle{}}\OperatorTok{)} \ControlFlowTok{continue}\OperatorTok{;}
\NormalTok{                res }\OperatorTok{+=}\NormalTok{ cal}\OperatorTok{(}\NormalTok{i }\OperatorTok{{-}} \DecValTok{1}\OperatorTok{,}\NormalTok{ sum }\OperatorTok{/} \DecValTok{10}\OperatorTok{);}
            \OperatorTok{\}}
    \ControlFlowTok{return}\NormalTok{ dp}\OperatorTok{[}\NormalTok{i}\OperatorTok{][}\NormalTok{carry}\OperatorTok{]} \OperatorTok{=}\NormalTok{ res}\OperatorTok{;}
\OperatorTok{\}}

\NormalTok{cal}\OperatorTok{(}\NormalTok{n}\OperatorTok{.}\NormalTok{size}\OperatorTok{(),} \DecValTok{0}\OperatorTok{)}
\end{Highlighting}
\end{Shaded}

Độ phức tạp sẽ còn \(O(10 * 100 * 10^3 * T) \approx 10^6 * T\) với
\(T \le 1000\) là được.

\end{document}
