% https://www.codechef.com/START105B?order=desc&sortBy=successful_submissions

\begin{center}
    {\LARGE \textbf{ĐỀ LUYỆN 01}}\\[6pt]
    {\large (Dành cho đội tuyển quốc gia)}
\end{center}

\section*{Bài 1. Dãy Spooky}

Ngày xưa, ở một miền đất xa xôi, một mụ phù thủy tinh quái lén theo dõi một nhóm người đang vui vẻ tụ họp.

Bị bao trùm bởi dục vọng đen tối, mụ quyết tâm chấm dứt những cuộc vui ấy và giết tất cả $N$ người. Mụ biết người thứ $i$ có \textbf{sức mạnh} $A_i$.
Mụ cũng biết có $M$ mối quan hệ \textbf{bạn bè} giữa các cặp người; quan hệ bạn bè có tính \textbf{bắc cầu}: nếu $X$ là bạn với $Y$ và $Y$ là bạn với $Z$ thì $X$ cũng là bạn với $Z$.

Mụ muốn giết tất cả theo một trình tự \textit{spooky}.

\medskip
Một dãy $S$ được gọi là \emph{spooky} nếu:
\begin{enumerate}[label=\arabic*)]
  \item $S$ là hoán vị của $\{1,2,\dots,N\}$ (tức gồm $N$ số nguyên phân biệt từ $1$ đến $N$);
  \item Với mọi $1\le i<j\le N$, nếu $S_i$ và $S_j$ là bạn thì phải có
  \[
    A_{S_i}\le A_{S_j}.
  \]
  (Trong hai người là bạn, kẻ có sức mạnh lớn hơn hẳn \textit{không được} đứng sớm hơn người kia.)
\end{enumerate}

\textbf{Yêu cầu.} Hãy đếm số dãy \emph{spooky}. Vì đáp án có thể rất lớn, in kết quả theo \(\mathbf{mod\ 10^9+7}\).

\subsubsection*{Dữ liệu vào}
\begin{itemize}
  \item Dòng đầu chứa số nguyên $T$ — số bộ test.
  \item Với mỗi bộ test:
  \begin{itemize}
    \item Dòng 1: hai số nguyên $N,\ M$ — số người và số quan hệ bạn bè.
    \item $M$ dòng tiếp theo: dòng thứ $i$ chứa hai số nguyên $u_i,\ v_i$ — một quan hệ bạn bè giữa $u_i$ và $v_i$.
    \item Dòng cuối: $N$ số nguyên $A_1, A_2, \dots, A_N$ — sức mạnh các người.
  \end{itemize}
\end{itemize}

\subsubsection*{Dữ liệu ra}
Với mỗi bộ test, in ra \textbf{một số nguyên} — số dãy \emph{spooky}, theo \(\,10^9+7\).

\subsubsection*{Ràng buộc}
\begin{itemize}
  \item \(1 \le T \le 2\cdot 10^4\)
  \item \(1 \le N \le 2\cdot 10^5\)
  \item \(0 \le M \le \min\!\big(2\cdot 10^5,\ \frac{N(N-1)}{2}\big)\)
  \item \(1 \le u_i, v_i \le N,\ u_i\neq v_i\);\quad mỗi cặp \((u_i,v_i)\) xuất hiện \textbf{không quá một lần}
  \item \(1 \le A_i \le 10^9\)
  \item Tổng \(N\) trên mọi test \(\le 2\cdot 10^5\);\quad tổng \(M\) trên mọi test \(\le 2\cdot 10^5\)
\end{itemize}

\subsubsection*{Ví dụ mẫu}

\begin{tcolorbox}[title=Input]
2\\
5 5\\
1 2\\
2 3\\
3 4\\
4 2\\
3 1\\
10 12 15 20 15\\
5 2\\
2 3\\
4 5\\
6 4 4 3 1
\end{tcolorbox}

\begin{tcolorbox}[title=Output]
5\\
60
\end{tcolorbox}

\textbf{Giải thích}

\textbf{Test 1:}  
Mỗi cặp trong $\{1,2,3,4\}$ đều là bạn của nhau, còn $5$ không là bạn với ai.  
Vì nhóm $\{1,2,3,4\}$ phải được sắp theo thứ tự tăng của sức mạnh, có 5 dãy spooky thỏa mãn:
\[
[5,1,2,3,4],\ [1,5,2,3,4],\ [1,2,5,3,4],\ [1,2,3,5,4],\ [1,2,3,4,5].
\]

\textbf{Test 2:}  
2 và 3 là bạn, 4 và 5 là bạn, còn 1 không bạn với ai.  
Vì $A_2 = A_3$ nên vị trí của chúng có thể hoán đổi, còn $A_4 > A_5$ nên 4 phải đứng sau 5.  
Không có ràng buộc nào khác, tổng cộng có $60$ dãy spooky.

\newpage

\section*{Bài 2. Alice và cây LCS}

Alice muốn tặng Bob một số phần thưởng, nhưng cô chưa biết nên cho bao nhiêu. Để quyết định, Alice đưa ra cho Bob một thử thách như sau:

Alice có một cây (đồ thị vô hướng liên thông gồm $N$ đỉnh và $N-1$ cạnh). Mỗi cạnh mang một ký tự chữ thường.  

Ngoài ra, Alice có một xâu $S$ độ dài $M$, gồm các chữ cái thường.

Gọi $str(u,v)$ là xâu ký tự thu được khi đi theo đường duy nhất giữa $u$ và $v$ trên cây.  
Alice yêu cầu Bob chọn hai đỉnh $u,v$, và sẽ nhận được số phần thưởng bằng:
\[
LCS(str(u,v), S)
\]
trong đó $LCS(A,B)$ là độ dài của xâu con chung dài nhất giữa $A$ và $B$.

Ví dụ:
\[
LCS("aba","cd")=0,\quad LCS("aba","aa")=2,\quad LCS("abc","cba")=1
\]

Hãy tìm \textbf{số phần thưởng tối đa} Bob có thể nhận được.

\subsubsection*{Dữ liệu vào}
\begin{itemize}
    \item Dòng đầu chứa số nguyên $T$.
    \item Mỗi test:
    \begin{itemize}
        \item Dòng 1: số nguyên $N$.
        \item $N-1$ dòng tiếp: mỗi dòng chứa $u_i, v_i, c_i$ (hai đỉnh và ký tự cạnh nối).
        \item Dòng cuối: xâu $S$.
    \end{itemize}
\end{itemize}

\subsubsection*{Dữ liệu ra}
Với mỗi test, in ra số nguyên — số phần thưởng tối đa.

\subsubsection*{Giới hạn}
\begin{itemize}
    \item $1 \le T \le 500$
    \item $2 \le N \le 10^4$
    \item $1 \le |S| \le 10^3$
    \item Tổng $N$ qua các test $\le 10^4$
    \item Tổng $|S|$ qua các test $\le 10^3$
\end{itemize}

\vspace{6pt}
\textbf{Ví dụ mẫu}

\begin{tcolorbox}[title=Input]
3\\
6\\
1 2 a\\
2 3 c\\
2 4 d\\
1 5 b\\
3 6 d\\
abcd\\
5\\
1 2 a\\
2 3 c\\
2 4 d\\
1 5 b\\
cfgd\\
3\\
1 3 a\\
2 3 a\\
b
\end{tcolorbox}

\begin{tcolorbox}[title=Output]
3\\
2\\
0
\end{tcolorbox}

\textbf{Giải thích}

\textbf{Test 1:}  
Bob có thể chọn $u=1$ và $v=6$.  
Đường đi $u \to v$ là $1 \to 2 \to 3 \to 6$, tương ứng với xâu ký tự `"acd"`.  
Khi đó:
\[
LCS("acd", "abcd") = 3,
\]
và đây là giá trị tốt nhất Bob có thể đạt được.

\textbf{Test 2:}  
Bob có thể chọn $u=3$ và $v=4$, tạo thành xâu `"cd"`.  
So sánh với $S="cfgd"$, ta có:
\[
LCS("cd", "cfgd") = 2.
\]
Vì vậy kết quả là $2$.

\newpage

\section*{Bài 3. Reaper và các ngôi nhà Halloween}

Vào lễ Halloween, Thần Chết (Grim Reaper) muốn thu hồn từ các ngôi nhà.  
Có $N$ ngôi nhà. Ngôi nhà thứ $i$ có \textbf{mức an toàn} $A_i$ và có $B_i$ người sinh sống.

Thần Chết có mức năng lực $L$, ban đầu $L = 0$.  
Khi Thần Chết ghé thăm một ngôi nhà, hai việc sau xảy ra theo thứ tự:

\begin{enumerate}[label=\arabic*.]
    \item Nếu $L \ge A_i$, Thần Chết sẽ thu hồn toàn bộ $B_i$ người ở nhà đó.  
          Nếu $L < A_i$ thì không thu được hồn nào.
    \item Sau đó, $L$ được gán bằng $A_i$, bất kể có thu hồn hay không.
\end{enumerate}

Thần Chết sẽ ghé \textbf{mỗi ngôi nhà đúng một lần}, theo một thứ tự nào đó.

Tuy nhiên, có $M$ ngôi nhà \textbf{đặc biệt} phải được ghé theo \textbf{thứ tự xác định}.
Thông tin này được cho bởi mảng $C = [C_1, C_2, \ldots, C_M]$ gồm $M$ chỉ số khác nhau, trong đó nhà $C_i$ phải được ghé trước nhà $C_{i+1}$.

\textbf{Hãy tìm số hồn tối đa} mà Thần Chết có thể thu được nếu chọn thứ tự ghé thăm các nhà một cách tối ưu và tuân thủ ràng buộc trên.

\subsubsection*{Dữ liệu vào}
\begin{itemize}
    \item Dòng đầu chứa số nguyên $T$ — số bộ test.
    \item Mỗi bộ test gồm:
    \begin{itemize}
        \item Dòng 1: hai số nguyên $N, M$.
        \item Dòng 2: $N$ số nguyên $A_1, A_2, \ldots, A_N$.
        \item Dòng 3: $N$ số nguyên $B_1, B_2, \ldots, B_N$.
        \item Dòng 4: $M$ số nguyên $C_1, C_2, \ldots, C_M$ (nếu $M=0$ thì dòng trống).
    \end{itemize}
\end{itemize}

\subsubsection*{Dữ liệu ra}
Với mỗi test, in ra một số nguyên — số hồn tối đa có thể thu được.

\subsubsection*{Giới hạn}
\begin{itemize}
    \item $1 \le T \le 10^5$
    \item $1 \le N \le 10^5$
    \item $0 \le M \le N$
    \item $1 \le A_i \le 10^5$
    \item $1 \le B_i \le 10^4$
    \item $1 \le C_i \le N$ (các $C_i$ đôi một khác nhau)
    \item Tổng $N$ qua các test $\le 3\cdot10^5$
\end{itemize}

\vspace{6pt}
\textbf{Ví dụ mẫu}

\begin{tcolorbox}[title=Input]
3\\
5 5\\
5 4 2 1 3\\
8 1 10 2 4\\
3 4 1 2 5\\
5 2\\
5 4 2 1 3\\
8 1 10 2 4\\
5 2\\
7 0\\
6 8 4 3 5 8 2\\
7 2 8 1 6 8 1
\end{tcolorbox}

\begin{tcolorbox}[title=Output]
7\\
16\\
31
\end{tcolorbox}

\textbf{Giải thích}

\textbf{Test 1:}  
Thần Chết chỉ có thể đi theo thứ tự $[3,4,1,2,5]$. Quá trình diễn ra như sau:
\begin{itemize}
    \item Ban đầu $L=0$.  
    \item Nhà 3: $A_3=2>0$, không thu được hồn nào, sau đó $L=2$.  
    \item Nhà 4: $A_4=1\le2$, thu được $B_4=2$ hồn, sau đó $L=1$.  
    \item Nhà 1: $A_1=5>1$, không thu được hồn nào, sau đó $L=5$.  
    \item Nhà 2: $A_2=4\le5$, thu được $B_2=1$ hồn, sau đó $L=4$.  
    \item Nhà 5: $A_5=3\le4$, thu được $B_5=4$ hồn, sau đó $L=3$.  
\end{itemize}

Tổng cộng $1+2+4=7$ hồn được thu.

\medskip
\textbf{Test 2:}  
Thần Chết có thể đi theo bất kỳ thứ tự nào miễn là nhà $5$ được ghé trước nhà $2$.  
Một cách tối ưu là đi theo $1 \to 5 \to 4 \to 2 \to 3$, thu được $10+2+4=16$ hồn.  
Không có cách nào đạt nhiều hơn thế.

\medskip
\textbf{Test 3:}  
Sau khi tối ưu mọi thứ tự hợp lệ, có thể chứng minh rằng không thể thu được nhiều hơn $31$ hồn.
