\begin{center}
    {\LARGE \textbf{ĐỀ LUYỆN 03}}\\[6pt]
    {\large (Dành cho đội tuyển quốc gia)}
\end{center}

\begin{tabularx}{\textwidth}{|c|X|c|c|c|}
\hline
\textbf{Bài} & \textbf{Tên bài} & \textbf{File đầu vào} & \textbf{File đầu ra} & \textbf{Điểm} \\
\hline
1 & Các đoạn con đẹp nhất & \texttt{BEAUSEQ.INP} & \texttt{BEAUSEQ.OUT} & 7 \\
\hline
2 & Công tắc và bóng đèn & \texttt{LTB.INP} & \texttt{LTB.OUT} & 7 \\
\hline
3 & Dãy con & \texttt{SUBSEQ.INP} & \texttt{SUBSEQ.OUT} & 6 \\
\hline
\end{tabularx}


\section*{Bài 1. Các đoạn con đẹp nhất}
% https://www.codechef.com/problems/CHSEG

Bạn được cho một mảng $A$ gồm $N$ số nguyên. Bạn được phép sắp xếp lại các phần tử của mảng theo bất kỳ thứ tự nào.

Một đoạn $[l, r]$ của mảng được gọi là \textit{đẹp} (beautiful) nếu, với mọi cặp $(i, j)$ sao cho $l \le i \le j \le r$, ta có:
$$
A_i \bmod A_j = 0 \quad \text{hoặc} \quad A_j \bmod A_i = 0.
$$

Giá trị \textit{điểm số} (score) của một đoạn đẹp $[l, r]$ được tính như sau:
$$
score([l,r]) = \min(A_l, A_{l+1}, \ldots, A_r) \times (r - l + 1)
$$

Nhiệm vụ của bạn là xác định \textbf{giá trị điểm số lớn nhất có thể} của một đoạn đẹp trong mảng $A$ sau khi sắp xếp lại các phần tử tùy ý.

\subsubsection*{Dữ liệu vào}

\begin{itemize}
    \item Dòng đầu tiên chứa một số nguyên $T$, là số lượng bộ test.  
    \item Mỗi bộ test gồm:
    \begin{itemize} 
        \item Dòng đầu tiên chứa một số nguyên $N$ — độ dài của mảng $A$.
        \item Dòng thứ hai chứa $N$ số nguyên $A_1, A_2, \ldots, A_N$ — các phần tử của mảng.
    \end{itemize}
\end{itemize}

\subsubsection*{Dữ liệu ra}

Với mỗi test, in ra trên một dòng giá trị điểm số lớn nhất có thể đạt được.

\subsubsection*{Ràng buộc}

\begin{itemize}
    \item $1 \le T \le 10^5$
    \item $1 \le N, A_i \le 10^5$
    \item Tổng $N$ trên tất cả các test không vượt quá $3 \times 10^5$
    \item Tổng của các giá trị $\max(A_i)$ trên tất cả các test không vượt quá $3 \times 10^5$
\end{itemize}

\subsubsection*{Ví dụ mẫu}

\begin{tcolorbox}[title=Input]
\begin{verbatim}
2
4
2 3 4 6
5
4 4 8 10 4
\end{verbatim}
\end{tcolorbox}

\begin{tcolorbox}[title=Output]
\begin{verbatim}
6
16
\end{verbatim}
\end{tcolorbox}

\textbf{Giải thích}

\textbf{Test 1:}  
Mảng có thể được sắp xếp lại thành $[3, 6, 2, 4]$.  
Các đoạn đẹp gồm:
\[
[1,1], [2,2], [3,3], [4,4], [1,2], [2,3], [3,4].
\]
Đoạn $[1,2]$ có:
\[
\min(A_1, A_2) \times 2 = 3 \times 2 = 6,
\]
là điểm số lớn nhất.

\textbf{Test 2:}  
Mảng có thể sắp xếp lại thành $[4, 4, 8, 4, 10]$.  
Đoạn đẹp $[1,4]$ cho điểm số:
\[
\min(A_1, A_2, A_3, A_4) \times 4 = 4 \times 4 = 16,
\]
là giá trị lớn nhất.

\newpage

\section*{Bài 2. Công tắc và bóng đèn}
% https://www.codechef.com/problems/LTB11121

Bạn có một bóng đèn và $N$ công tắc. Bóng đèn chỉ sáng nếu trạng thái $N$ công tắc đúng theo xâu nhị phân $S$ độ dài $N$. Ở xâu này, ký tự $0$ tại vị trí $i$ nghĩa là công tắc thứ $i$ TẮT, ký tự $1$ nghĩa là công tắc thứ $i$ BẬT.

Ban đầu, trạng thái các công tắc được cho bởi xâu nhị phân $T$ độ dài $N$.

Bạn \textbf{phải} thực hiện \textbf{chính xác hai} lần thao tác sau để làm bóng đèn sáng:

Chọn một đoạn con $T[l,r]$ với $1\le l\le r\le N$, và \emph{lật} tất cả ký tự trong đoạn đó (tức là đổi $0 \leftrightarrow 1$).

Hai dãy thao tác $T[a_1,a_2]$ rồi $T[a_3,a_4]$ và $T[b_1,b_2]$ rồi $T[b_3,b_4]$ được coi là \textit{khác nhau} nếu tồn tại $k\in\{1,2,3,4\}$ sao cho $a_k\ne b_k$.

Nhiệm vụ của bạn là đếm số cách thực hiện đúng hai thao tác để sau cùng $T$ trở thành $S$ (bóng đèn sáng).

\subsubsection*{Dữ liệu vào}
Dòng đầu chứa số nguyên $T$ — số lượng bộ test.  

Mỗi bộ test gồm nhiều dòng:
\begin{itemize}
  \item Dòng đầu chứa một số nguyên $N$ — số lượng công tắc.
  \item Dòng thứ hai chứa xâu nhị phân $S$ độ dài $N$ — trạng thái yêu cầu.
  \item Dòng thứ ba chứa xâu nhị phân $T$ độ dài $N$ — trạng thái ban đầu.
\end{itemize}

\subsubsection*{Dữ liệu ra}
Với mỗi bộ test, in ra một số nguyên trên một dòng — số cách thực hiện \textbf{hai} thao tác lật đoạn để biến $T$ thành $S$.

\subsubsection*{Ràng buộc}
\begin{itemize}
  \item $1 \le T \le 2\cdot 10^5$
  \item $1 \le N \le 10^6$
  \item $S$ và $T$ chỉ gồm các ký tự \texttt{0} và \texttt{1}.
  \item Tổng $N$ trên tất cả các bộ test không vượt quá $10^6$.
\end{itemize}

\subsubsection*{Ví dụ mẫu}

\begin{tcolorbox}[title=Input]
\begin{verbatim}
2
4
0111
1010
3
010
111
\end{verbatim}
\end{tcolorbox}

\begin{tcolorbox}[title=Output]
\begin{verbatim}
6
6
\end{verbatim}
\end{tcolorbox}

\textbf{Giải thích}

\textbf{Test 1.} Cần đạt $S=\texttt{0111}$ từ $T=\texttt{1010}$ bằng đúng hai lần lật đoạn. Có 6 cách, ví dụ:
\[
[1,2]\ \text{rồi}\ [4,4];\quad
[4,4]\ \text{rồi}\ [1,2];\quad
[1,3]\ \text{rồi}\ [3,4];\quad
[3,4]\ \text{rồi}\ [1,3];\quad
[3,3]\ \text{rồi}\ [1,4];\quad
[1,4]\ \text{rồi}\ [3,3].
\]
Mỗi cặp trên biểu diễn hai đoạn con của $T$ được lật theo thứ tự tương ứng và sau cùng thu được $S$.

\newpage

\section*{Bài 3: Dãy con}
% https://www.codechef.com/problems/CT27

Bạn được cho hai mảng số nguyên $A$ và $B$ có độ dài lần lượt là $N$ và $M$, cùng một mảng rỗng $C$. Trong một thao tác, bạn có thể:
\begin{itemize}
  \item Nối \emph{toàn bộ} mảng $A$ vào cuối $C$; \textbf{hoặc}
  \item Tăng \textbf{một} phần tử bất kỳ của $C$ lên $1$.
\end{itemize}
Hãy tìm số thao tác tối thiểu để $B$ trở thành một \textit{dãy con} (subsequence) của $C$, hoặc in ra $-1$ nếu không thể đạt được.

\subsubsection*{Dữ liệu vào}
Dòng đầu chứa số nguyên $T$ — số lượng bộ test.  

Mỗi bộ test gồm ba dòng:
\begin{itemize}
  \item Dòng đầu: hai số nguyên $N, M$ — độ dài các mảng $A$ và $B$.
  \item Dòng thứ hai: $N$ số nguyên $A_1, A_2, \ldots, A_N$.
  \item Dòng thứ ba: $M$ số nguyên $B_1, B_2, \ldots, B_M$.
\end{itemize}

\subsubsection*{Dữ liệu ra}
Với mỗi bộ test, in một số nguyên — số thao tác tối thiểu để $B$ trở thành dãy con của $C$, hoặc $-1$ nếu không thể.

\subsubsection*{Ràng buộc}
\begin{itemize}
  \item $1 \le T \le 10^5$
  \item $1 \le N \le 2\cdot 10^5$, \quad $1 \le M \le 2\cdot 10^5$
  \item $1 \le A_i, B_i \le 10^9$
  \item Tổng $N$ trên tất cả các test $\le 2\cdot 10^5$.
  \item Tổng $M$ trên tất cả các test $\le 2\cdot 10^5$.
\end{itemize}

\subsubsection*{Ví dụ mẫu}

\begin{tcolorbox}[title=Input]
\begin{verbatim}
4
2 3
5 2
4 5 3
1 4
10
11 15 14 10
5 3
3 9 10 5 5
8 1 2
4 3
1 3 1 2
1 1 2
\end{verbatim}
\end{tcolorbox}

\begin{tcolorbox}[title=Output]
\begin{verbatim}
5
14
-1
1
\end{verbatim}
\end{tcolorbox}

\textbf{Giải thích}

\textbf{Test 1.}  
Khởi đầu $C=[\,]$.  
Nối $A$ hai lần $\Rightarrow C=[5,2,5,2]$.  
Tăng $C_2$ hai lần và $C_4$ một lần để được $C=[5,4,5,3]$. Khi đó $B=[4,5,3]$ là dãy con của $C$. Tổng $5$ thao tác.

\textbf{Test 2.}  
Nối $A$ bốn lần để có $[10,10,10,10]$, rồi tăng từng phần tử tương ứng để đạt $[11,15,14,10]$. Tổng $14$ thao tác.

\textbf{Test 3.}  
Không thể để $B$ xuất hiện như một dãy con của $C$ trong mọi cách làm.

\textbf{Test 4.}  
Nối $A$ một lần là đủ vì $B$ đã là dãy con của $A$.
