\begin{center}
    {\LARGE \textbf{LỜI GIẢI ĐỀ LUYỆN 02}}\\[6pt]
    {\large (Dành cho đội tuyển quốc gia)}
\end{center}

\section*{Bài 1. Thứ tự từ điển lớn nhất}

\textbf{Nhận xét:}  
Vì \(\gcd(B_1, B_2, \ldots, B_i)\) luôn là bội của \(\gcd(B_1, B_2, \ldots, B_{i+1})\), nên để tồn tại mảng \(B\), dãy \(A\) phải thỏa mãn điều kiện  
\[
A_i \text{ là ước của } A_{i-1} \quad \forall i > 1.
\]

\textbf{Bước đầu:}  
Ta có duy nhất một lựa chọn cho \(B_1\):  
\[
B_1 = A_1,
\]
vì đây là cách duy nhất để \(\gcd(B_1) = A_1\).

\textbf{Bước xây dựng tham lam:}  
Giả sử đã biết \(B_1, B_2, \ldots, B_{i-1}\) sao cho \(\gcd(B_1, B_2, \ldots, B_{i-1}) = A_{i-1}\).  
Khi đó, ta cần chọn \(B_i\) sao cho:
\[
\gcd(B_1, B_2, \ldots, B_i) = A_i.
\]
Do \(\gcd(B_1, \ldots, B_{i-1}) = A_{i-1}\), điều này tương đương với:
\[
\gcd(A_{i-1}, B_i) = A_i.
\]

Vì \(A_i\) là ước của \(A_{i-1}\), đặt \(P = \dfrac{A_{i-1}}{A_i}\).  
Ta cần \(B_i\) thỏa mãn:
\[
B_i \text{ là bội của } A_i \quad \text{và} \quad \gcd\!\left(\dfrac{B_i}{A_i},\, P\right) = 1.
\]

\textbf{Cách chọn \(B_i\):}  
Để mảng \(B\) có thứ tự từ điển lớn nhất, ta chọn \(B_i\) càng lớn càng tốt, tức là:
\[
B_i = A_i \cdot \left\lfloor \frac{M}{A_i} \right\rfloor,
\]
nghĩa là bội lớn nhất của \(A_i\) không vượt quá \(M\).

Tuy nhiên, giá trị này có thể không thỏa điều kiện \(\gcd(B_i, A_{i-1}) = A_i\).  
Khi đó, ta giảm \(B_i\) theo bước \(A_i\) cho đến khi đạt được:
\[
\text{Trong khi } \gcd(B_i, A_{i-1}) \ne A_i, \text{ thì } B_i \leftarrow B_i - A_i.
\]

Khi vòng lặp dừng, ta thu được \(B_i\) lớn nhất có thể thỏa mãn điều kiện, tức là giá trị tối ưu về mặt thứ tự từ điển.

\textbf{Tính đúng đắn:}  
\begin{itemize}
    \item Nếu \(\gcd(A_{i-1}, B_i) \ne A_i\), khi đó \(\gcd(B_1, \ldots, B_i)\) không thể bằng \(A_i\), nên không hợp lệ.  
    \item Nếu \(\gcd(A_{i-1}, B_i) = A_i\), thì \(\gcd(B_1, \ldots, B_i) = A_i\) đúng như yêu cầu.  
    \item Quá trình chọn bội lớn nhất rồi giảm dần đảm bảo \(B\) có giá trị lớn nhất theo thứ tự từ điển.
\end{itemize}

\textbf{Độ phức tạp:}  
Mỗi bước chỉ cần vài phép tính \(\gcd\), và số lần giảm thường rất nhỏ (vì xác suất hai số ngẫu nhiên không chia hết nhau cao).  
Do đó, thuật toán chạy rất nhanh và hoàn toàn đáp ứng được giới hạn của bài toán.

\newpage

\section*{Bài 2. Đèn chùm}

\textbf{Phân tích lại vấn đề:}  
Ta cần xác định xem có tồn tại \emph{xác suất dương} (nghĩa là chỉ cần một kịch bản khả thi) sao cho \textbf{tất cả các giá đỡ đều sụp đổ}.  
Khi một giá đỡ $i$ sụp, trọng lượng $W_i$ của nó được phân bố ngẫu nhiên thành $x$ và $y$ sang hai bên $(i-1)$ và $(i+1)$.  
Tuy nhiên, vì ta chỉ cần biết có tồn tại \emph{một cách phân bố} khả thi hay không, nên yếu tố ngẫu nhiên này có thể được loại bỏ.  
Nói cách khác, ta hoàn toàn được phép chọn $x$ và $y$ tùy ý miễn sao tổng $x+y=W_i$.

\textbf{Quan sát quan trọng:}  
Khi giá đỡ $i$ sụp đổ, hai phía của nó (phía trái và phía phải) trở nên độc lập với nhau:
\begin{itemize}
    \item Phía trái gồm các giá đỡ $(1,2,\ldots,i-1)$;  
    \item Phía phải gồm $(i+1,i+2,\ldots,N)$.
\end{itemize}

Sau khi $i$ bị phá hủy, trọng lượng từ phía trái chỉ có thể lan dần về $1$, và tương tự, phía phải lan về $N$.  
Như vậy, việc phá hủy toàn bộ hệ thống có thể tách thành hai bài toán con: phá hủy toàn bộ dãy bên trái và bên phải của một điểm trung tâm.

\textbf{Định nghĩa:}  
Ký hiệu:
\begin{itemize}
    \item $\text{pref}[i]$ là lượng trọng lượng \textbf{nhỏ nhất cần thêm} vào giá đỡ thứ $i$ để tất cả các giá đỡ từ $1$ đến $i$ đều sụp đổ (giả sử $i+1$ không tồn tại).
    \item Tương tự, $\text{suf}[i]$ là lượng trọng lượng nhỏ nhất cần thêm vào giá đỡ thứ $i$ để tất cả các giá đỡ từ $i$ đến $N$ đều sụp đổ (giả sử $i-1$ không tồn tại). 
\end{itemize}

\textbf{Công thức quy hoạch động cho tiền tố:}  
Đầu tiên, để giá đỡ $i$ sụp, ta cần ít nhất:
\[
\text{pref}[i] \ge A_i - W_i,
\]
vì $W_i$ ban đầu chưa đủ để phá hủy nó.

Ngoài ra, sau khi $i$ sụp, ta phải đảm bảo toàn bộ phần trước $(1,\ldots,i-1)$ cũng sụp.  
Theo định nghĩa, điều này cần ít nhất $\text{pref}[i-1]$ trọng lượng bổ sung.  
Trọng lượng này được truyền từ $i$ sang, nên tại $i$ ta cần có ít nhất $\text{pref}[i-1] - W_i$ bổ sung.

Từ hai điều kiện trên, ta có công thức gộp:
\[
\boxed{\text{pref}[i] = \max(A_i,\, \text{pref}[i-1]) - W_i.}
\]

\textbf{Tương tự cho hậu tố:}
\[
\boxed{\text{suf}[i] = \max(A_i,\, \text{suf}[i+1]) - W_i.}
\]

\textbf{Tính đáp án cho mỗi vị trí $i$:}  
Khi ta chọn thêm trọng lượng vào giá đỡ $i$, ta cần đảm bảo:
\begin{itemize}
  \item Giá đỡ $i$ đủ để nó sụp;
  \item Phần bên trái $(1,\ldots,i-1)$ đủ để sụp hoàn toàn;
  \item Phần bên phải $(i+1,\ldots,N)$ đủ để sụp hoàn toàn.
\end{itemize}

Tổng yêu cầu tối thiểu là:
\[
\max(A_i,\, \text{pref}[i-1] + \text{suf}[i+1]).
\]
Vì giá đỡ $i$ đã có sẵn $W_i$, lượng trọng lượng cần thêm vào nó sẽ là:
\[
\boxed{\text{ans}[i] = \max(A_i,\, \text{pref}[i-1] + \text{suf}[i+1]) - W_i.}
\]

\textbf{Tóm tắt quy trình:}
\begin{enumerate}
  \item Tính mảng \(\text{pref}[i]\) từ trái sang phải bằng công thức:  
        \(\text{pref}[i] = \max(A_i,\, \text{pref}[i-1]) - W_i\), với \(\text{pref}[0] = 0\).
  \item Tính mảng \(\text{suf}[i]\) từ phải sang trái:  
        \(\text{suf}[i] = \max(A_i,\, \text{suf}[i+1]) - W_i\), với \(\text{suf}[N+1] = 0\).
  \item Cuối cùng, với mỗi \(i\), tính:  
        \(\text{ans}[i] = \max(A_i,\, \text{pref}[i-1] + \text{suf}[i+1]) - W_i.\)
\end{enumerate}

\textbf{Diễn giải ý nghĩa:}  
Giá trị \(\text{ans}[i]\) chính là lượng trọng lượng nhỏ nhất cần thêm vào giá đỡ thứ \(i\) để \emph{tồn tại một chuỗi sụp đổ toàn bộ hệ thống} — nghĩa là có ít nhất một kịch bản khả thi khiến toàn bộ các giá đỡ đều bị phá hủy.


\newpage

\section*{Bài 3. Dãy chuẩn bình phương}

Trước hết, hãy phân tích cấu trúc của một dãy \(B\) hợp lệ:
\begin{itemize}
    \item Điều kiện "độ dài lẻ" yêu cầu \emph{mọi} dãy con độ dài lẻ của \(B\) đều \textbf{không} có tích là số chính phương; từ đó suy ra \emph{mỗi phần tử} của \(B\) đều \textbf{không} là số chính phương.
    \item Điều kiện "độ dài chẵn" yêu cầu \emph{mọi cặp} phần tử trong \(B\) phải có tích là số chính phương.
\end{itemize}

Hơn nữa, hai điều kiện này là \textbf{đủ} để đảm bảo rằng toàn bộ dãy con đều hợp lệ, bởi vì:
\begin{itemize}
  \item Mọi dãy con độ dài chẵn đều có thể chia thành các cặp; tích của mỗi cặp là chính phương, nên tích toàn bộ cũng là chính phương.
  \item Tích của một số chính phương với một số \emph{không} chính phương luôn \emph{không} là chính phương; vì vậy mọi dãy con độ dài lẻ (gồm một số lẻ cặp chính phương và dư ra một phần tử đơn lẻ không chính phương) đều thỏa yêu cầu.
\end{itemize}

\textbf{Giảm bài toán.}
Trước hết, loại bỏ mọi số \emph{chính phương} khỏi \(A\) (vì chúng không thể nằm trong \(B\)).
Vấn đề còn lại: với hai số nguyên dương \(x, y\), khi nào \(x\cdot y\) là số chính phương?

\textbf{Trả lời.}
Viết phân tích thừa số nguyên tố:
\[
x = \prod_{i=1}^{k} p_i^{a_i},\qquad
y = \prod_{i=1}^{k} p_i^{b_i}.
\]
Khi đó
\[
x\cdot y = \prod_{i=1}^{k} p_i^{a_i+b_i}
\]
là \textbf{chính phương} khi và chỉ khi \(\forall i\), \(a_i+b_i\) \emph{chẵn}, tức là \(a_i\) và \(b_i\) có \emph{cùng} tính chẵn lẻ.
Nói cách khác, \(x\) và \(y\) phải có \textbf{cùng tập các thừa số nguyên tố xuất hiện với số mũ lẻ}.
Gọi \(x_s\) (tương ứng \(y_s\)) là \emph{phần squarefree} của \(x\) (tức là tích các thừa số nguyên tố của \(x\) có mũ lẻ), điều kiện tương đương với
\[
x_s = y_s.
\]

\textbf{Hệ quả.}
Hai phần tử có thể cùng xuất hiện trong dãy con \(B\) khi và chỉ khi \emph{phần squarefree} của chúng \emph{giống nhau}.
Do đó, \emph{mọi} phần tử của dãy con \(B\) phải có \emph{cùng} squarefree part.
Bài toán trở thành: \emph{thay} mỗi phần tử \(A_i\) bởi squarefree part của nó rồi \textbf{đếm bội số xuất hiện lớn nhất} của một giá trị — đó chính là độ dài dãy \(B\) dài nhất.

\textbf{Tính squarefree part nhanh.}
Ta cần phân tích thừa số nhanh đến \(M=10^7\):
\begin{enumerate}
  \item Tiền xử lý bằng sàng: với mọi \(x \le M\), lưu một thừa số nguyên tố nhỏ nhất \(\mathrm{prm}[x]\).
  \item Để phân tích \(x\): lặp
  \begin{itemize}
    \item Nếu \(x=1\) thì dừng.
    \item Lấy \(p=\mathrm{prm}[x]\), đếm bội số mũ của \(p\) trong \(x\) bằng cách chia nhiều lần.
    \item Nếu số mũ của \(p\) là \textbf{lẻ}, nhân \(p\) vào squarefree part.
  \end{itemize}
\end{enumerate}

\textbf{Thuật toán tổng quát.}
\begin{enumerate}
  \item Loại các phần tử là \emph{chính phương} khỏi \(A\).
  \item Với mỗi phần tử còn lại, tính squarefree part (theo sàng ở trên).
  \item Tần suất—đếm số lần xuất hiện của từng squarefree part; đáp án là \textbf{giá trị tần suất lớn nhất}.
\end{enumerate}

\textbf{Độ phức tạp.}
Tiền xử lý sàng: \(\mathcal{O}(M \log\log M)\) với \(M=10^7\).  
Mỗi test: \(\mathcal{O}(N \log M)\) (do mỗi lần “tuột” theo thừa số nhỏ nhất chia được).
