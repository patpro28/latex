% !TEX program = xelatex
\documentclass[aspectratio=169,11pt]{beamer}

\usepackage{tikz}
\usetikzlibrary{graphs, positioning}
% pdfLaTeX fallback
\usepackage[utf8]{inputenc}
\usepackage[T5]{fontenc}
\usepackage[vietnamese]{babel}


% ==== Beamer theme & common packages ====
\usetheme{Madrid}
\usecolortheme{seagull}
\setbeamertemplate{navigation symbols}{}
\setbeamertemplate{footline}[frame number]
\usepackage{amsmath,amssymb}
\usepackage{booktabs}

\title{Hint 9C5}
\author{TMATH EDU}
\date{\today}
\begin{document}
\begin{frame}[plain]\titlepage\end{frame}

\begin{frame}{Buổi 1 - Ước, Bội}
\begin{itemize}
  \item \textbf{Bài 11}: Tìm ước lẻ lớn nhất của $n$, thực hiện loại bỏ thừa số $2$ trong $n$.
  \item \textbf{Bài 12}: Tìm giá trị $g$ lớn nhất mà có ít nhất $2$ số chia hết cho nó, khi đó $g$ chính là kết quả. Sử dụng mảng thống kê kết hợp thuật toán sàng ước để đếm số lượng số chia hết cho $g$.
  \item \textbf{Bài 13}: Áp dụng công thức $a\times b = g \times l$, chia hai vế cho $g^2$ ta có, $\frac{a}{g}\times\frac{b}{g} = \frac{l}{g}$, từ đó duyệt ước của $\frac{l}{g}$.
  \item \textbf{Bài 14}: Theo công thức Euclid, $gcd(a,b)$ là ước của $|b-a|$. Ta nhận thấy $gcd(a-k.g, b-k.g)$ là bội của $gcd(a,b)$ với $k$ là số nguyên. Dựa vào 2 tính chất này, ta có thể tìm ra được $gcd$ kế tiếp sau mỗi bước, từ đó tính ra số bước.
\end{itemize}
\end{frame}

\begin{frame}{Buổi 2 - Số nguyên tố}
\begin{itemize}
  \item \textbf{Bài 11}: 
  \item \textbf{Chủ nghĩa duy tâm}: Ý thức có trước, quyết định vật chất.
  \item \textbf{Nhất nguyên luận}: Thừa nhận một bản nguyên.
  \item \textbf{Nhị nguyên luận}: Thừa nhận hai bản nguyên vật chất và tinh thần.
  \item \textbf{Thuyết khả tri}: Con người có thể nhận thức thế giới.
  \item \textbf{Thuyết bất khả tri}: Phủ nhận khả năng nhận thức bản chất thế giới.
  \item \textbf{Hoài nghi luận}: Nghi ngờ tính chắc chắn của tri thức.
  \item \textbf{Phép biện chứng}: Nhìn sự vật trong mối liên hệ và vận động.
  \item \textbf{Phép siêu hình}: Nhìn sự vật cô lập, bất biến.
  \item \textbf{Biện chứng tự phát}: Hình thức sơ khai thời cổ đại.
  \item \textbf{Biện chứng duy tâm}: Vận động bắt nguồn từ tinh thần.
  \item \textbf{Biện chứng duy vật}: Vận động bắt nguồn từ mâu thuẫn nội tại của vật chất.
\end{itemize}
\end{frame}

\begin{frame}{Phạm trù cơ bản và Duy vật lịch sử}
\begin{itemize}
  \item \textbf{Vật chất}: Thực tại khách quan tồn tại độc lập ý thức.
  \item \textbf{Ý thức}: Sự phản ánh thế giới vào bộ óc, có tính sáng tạo.
  \item \textbf{Không gian}: Hình thức tồn tại về sắp xếp, vị trí, hình dạng.
  \item \textbf{Thời gian}: Hình thức tồn tại về trình tự, chiều hướng biến đổi.
  \item \textbf{Vận động}: Phương thức tồn tại của vật chất.
  \item \textbf{Phản ánh}: Sự tái tạo đặc điểm sự vật.
  \item \textbf{Lực lượng sản xuất}: Yếu tố vật chất và con người tạo ra sản phẩm.
  \item \textbf{Quan hệ sản xuất}: Quan hệ giữa người với người trong sản xuất.
  \item \textbf{Phương thức sản xuất}: Thống nhất lực lượng và quan hệ sản xuất.
  \item \textbf{Cơ sở hạ tầng}: Nền tảng kinh tế của xã hội.
  \item \textbf{Kiến trúc thượng tầng}: Hệ thống chính trị, pháp luật, tư tưởng.
  \item \textbf{Giai cấp}: Tập đoàn người có vị trí khác nhau trong sản xuất.
  \item \textbf{Đấu tranh giai cấp}: Mâu thuẫn giữa các giai cấp đối lập.
  \item \textbf{Nhà nước}: Bộ máy quyền lực của giai cấp thống trị.
  \item \textbf{Cách mạng xã hội}: Thay thế phương thức sản xuất cũ.
  \item \textbf{Tồn tại xã hội}: Điều kiện sinh hoạt vật chất.
  \item \textbf{Ý thức xã hội}: Quan điểm, tư tưởng của xã hội.
\end{itemize}
\end{frame}

\begin{frame}{Quan niệm về con người}
\begin{itemize}
  \item \textbf{Bản chất con người}: Tổng hòa các mối quan hệ xã hội.
  \item \textbf{Quan hệ xã hội}: Sự liên kết giữa người với người.
  \item \textbf{Phát triển toàn diện con người}: Hoàn thiện thể chất, trí tuệ, tinh thần.
\end{itemize}
\end{frame}

\begin{frame}{Lịch sử phát triển triết học}
\begin{itemize}
  \item \textbf{Triết học cổ đại}: Ra đời ở phương Đông và phương Tây (thế kỷ VIII - VI TCN), gắn với nhu cầu nhận thức và xã hội có giai cấp.
  \item \textbf{Triết học trung cổ}: Bị chi phối mạnh bởi tôn giáo, đặc biệt là thần học.
  \item \textbf{Triết học Phục hưng và Cận đại}: Gắn với sự phát triển khoa học tự nhiên, đề cao lý trí, con người.
  \item \textbf{Triết học cổ điển Đức}: Đỉnh cao của phép biện chứng duy tâm (Cantơ, Hêghen) và duy vật (Phơbach).
  \item \textbf{Triết học Mác - Lênin}: Cuộc cách mạng trong triết học, thống nhất chủ nghĩa duy vật với phép biện chứng, xây dựng chủ nghĩa duy vật lịch sử.
  \item \textbf{Triết học hiện đại}: Nhiều trào lưu đa dạng, nhưng triết học Mác - Lênin vẫn giữ vai trò nền tảng khoa học và phương pháp luận.
\end{itemize}
\end{frame}

\end{document}
