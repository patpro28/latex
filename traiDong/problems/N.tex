\section{Chia dãy k}\label{trai-dong-2024-n}

\subtitle{Time limit: \textbf{2 giây} \hfill Memory limit: \textbf{512}Mb}

\subsection{Đề bài}\label{trai-dong-2024-n-de-bai}

Cho dãy $a_1, a_2, ..., a_n$. Hãy chia dãy thành $k$ đoạn con liên
tiếp sao cho tổng độ đẹp của các đoạn là lớn nhất. Độ đẹp của một đoạn
là giá trị của phần tử nhỏ nhất thuộc đoạn đó.

\subsection{Dữ liệu vào}\label{trai-dong-2024-n-du-lieu-vao}

Dòng đầu chứa số $n, k$
$(1\le k\le n\le 10^5; n\times k\le 5\cdot 10^7)$

Dòng thứ hai chứa dãy $a_i$ $(0\le a_i\le 10^9)$

\subsection{Dữ liệu ra}\label{trai-dong-2024-n-du-lieu-ra}

Ghi ra tổng độ đẹp lớn nhất.

\subsection{Tính điểm}\label{trai-dong-2024-n-tinh-diem}

\begin{enumerate}
\def\labelenumi{\arabic{enumi}.}
\tightlist
\item
  $(15p)$: $k \le 2$
\item
  $(25p)$: $n\le 16$
\item
  $(25p)$: $n\le 500$
\item
  $(35p)$: không có ràng buộc gì thêm.
\end{enumerate}

\subsubsection{Sample Input 1}\label{trai-dong-2024-n-sample-input-1}

\begin{tcolorbox}
\begin{verbatim}
5 2
4 5 2 6 3
\end{verbatim}
\end{tcolorbox}

\subsubsection{Sample Output 1}\label{trai-dong-2024-n-sample-output-1}

\begin{tcolorbox}
\begin{verbatim}
6
\end{verbatim}
\end{tcolorbox}

\subsubsection{Sample Input 2}\label{trai-dong-2024-n-sample-input-2}

\begin{tcolorbox}
\begin{verbatim}
6 4
5 3 7 2 4 6
\end{verbatim}
\end{tcolorbox}

\subsubsection{Sample Output 2}\label{trai-dong-2024-n-sample-output-2}

\begin{tcolorbox}
\begin{verbatim}
18
\end{verbatim}
\end{tcolorbox}
