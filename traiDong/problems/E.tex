\section{Di chuyển về nhà}\label{trai-dong-2024-e}

\subtitle{Time limit: \textbf{1.5 giây} \hfill Memory limit: \textbf{512}Mb}

\subsection{Đề bài}\label{trai-dong-2024-e-de-bai}

Tại thành phố X, hệ thống giao thông rất đặc biệt, có tất cả $n$ giao
lộ được kết nối với nhau bởi $n-1$ con đường hai chiều, hệ thống giao
thông đảm bảo tất cả các giao lộ có thể di chuyển được đến nhau. Có
$k$ người đang đứng tại $k$ giao lộ khác nhau và cần di chuyển về
nhà, biết rằng người $i$ đứng tại giao lộ $s_i$ và cần di chuyển về
nhà nằm tại giao lộ $t_i$. Vì nhiều lý do khác nhau, thành phố X cần
sắp xếp để từng người di chuyển về nhà sao cho không có người nào di
chuyển trên đường và giao lộ mà gặp người khác. Khi một người đã đi đến
giao lộ nhà mình, người đó sẽ vào nhà và khi người khác đi qua giao lộ
này sẽ không gặp người này nữa.

Hãy tìm thứ tự di chuyển để đảm bảo không có hai người nào gặp nhau.

\subsection{Dữ liệu vào}\label{trai-dong-2024-e-du-lieu-vao}

Dòng đầu chứa hai số nguyên $n, k$ $(1\le k\le n\le 5\times 10^5)$

$n- 1$ dòng tiếp theo, mỗi dòng chứa hai số $u, v$
$(1\le u, v\le n)$ mô tả giao lộ $u$ và giao lộ $v$ có con đường
kết nối trực tiếp với nhau

$k$ dòng tiếp theo, mỗi dòng chứa hai số $s_i, t_i$
$(1\le s_i, t_i\le n; s_i\neq s_j \forall 1\le i\neq j\le k)$.

\subsection{Dữ liệu ra}\label{trai-dong-2024-e-du-lieu-ra}

Nếu tồn tại thứ tự di chuyển thì in ra một dòng gồm $k$ số tương ứng
là thứ tự di chuyển của $k$ người, ngược lại in ra $-1$.

\subsection{Tính điểm}\label{trai-dong-2024-e-tinh-diem}

\begin{enumerate}
\def\labelenumi{\arabic{enumi}.}
\tightlist
\item
  $(10\%)$: $k\le 10; n\le 100$
\item
  $(20\%)$: $n\le 100$
\item
  $(20\%)$: $n\le 5000$
\item
  $(25\%)$: $n\le 10^5$
\item
  $(25\%)$: không có ràng buộc gì thêm
\end{enumerate}

\subsubsection{Sample Input 1}\label{trai-dong-2024-e-sample-input-1}

\begin{tcolorbox}
\begin{verbatim}
5 4
1 2
2 3
4 2
4 5
1 3
3 2
5 3
4 3
\end{verbatim}
\end{tcolorbox}

\subsubsection{Sample Output 1}\label{trai-dong-2024-e-sample-output-1}

\begin{tcolorbox}
\begin{verbatim}
2 1 4 3
\end{verbatim}
\end{tcolorbox}

\subsubsection{Sample Input 2}\label{trai-dong-2024-e-sample-input-2}

\begin{tcolorbox}
\begin{verbatim}
4 2
1 2
2 3
3 4
3 1
2 4
\end{verbatim}
\end{tcolorbox}

\subsubsection{Sample Output 2}\label{trai-dong-2024-e-sample-output-2}

\begin{tcolorbox}
\begin{verbatim}
-1
\end{verbatim}
\end{tcolorbox}
