\section{Bài toán cái túi 2}\label{trai-dong-2024-f}

\subtitle{Time limit: \textbf{2 giây} \hfill Memory limit: \textbf{512}Mb}

\subsection{Đề bài}\label{trai-dong-2024-f-de-bai}

Có $n$ đồ vật được đánh số từ $1$ đến $n$, đồ vật $i$ có trọng
lượng $w_i$ và có giá trị là $v_i$. An có một cái túi có thể chứa
được các đồ vật có tổng khối lượng không quá $k$. Vì số lượng đồ vật
và trọng lượng các đồ vật là rất lớn do đó, nên để tìm được cách chọn đồ
vật có tổng khối lượng không vượt quá $k$ và có tổng giá trị là lớn
nhất, An thực hiện một cách như sau:

\begin{itemize}
\tightlist
\item
  Bỏ qua $t$ đồ vật đầu tiên, hay bỏ qua các đồ vật có chỉ số từ $1$
  đến $t$;
\item
  Xét lần lượt các đồ vật từ $t + 1$ đến $n$, xét đến đồ vật $i$,
  nếu trọng lượng của đồ vật $i$ khi thêm vào túi mà không làm quá sức
  chứa của túi thì sẽ thêm đồ vật $i$ và túi, ngược lại bỏ qua đồ vật
  $i$ và xét đến đồ vật tiếp theo.
\end{itemize}

Vì không biết giá trị $t$ nào mà với cách làm trên cho phép An chọn
được các đồ vật có tổng giá trị lớn nhất, do đó An quyết định tính toán
giá trị của các đồ vật lấy được với $t$ từ $0$ đến $n - 1$. Hãy
giúp An tìm các giá trị này.

\subsection{Dữ liệu vào}\label{trai-dong-2024-f-du-lieu-vao}

Dòng đầu chứa hai số nguyên $n, k$
$(1\le n \le 2\times 10^5; 1\le k\le 10^9)$

Dòng thứ hai chứa $n$ số nguyên $v_i$ $(1\le v_i\le 10^9)$

Dòng thứ ba chứa $n$ số nguyên $w_i$ $(1\le w_i\le 10^9)$

\subsection{Dữ liệu ra}\label{trai-dong-2024-f-du-lieu-ra}

Ghi ra $n$ số nguyên tương ứng là tổng giá trị của các đồ vật lấy được
với $t$ lần lượt từ $0$ đến $n-1$.

\subsection{Tính điểm}\label{trai-dong-2024-f-tinh-diem}

\begin{enumerate}
\def\labelenumi{\arabic{enumi}.}
\tightlist
\item
  $(15\%)$: $n\le 1000$
\item
  $(20\%)$: $k\le 100$
\item
  $(20\%)$: $w_i\le w_{i+1}$
\item
  $(45\%)$: không có ràng buộc gì thêm
\end{enumerate}

\subsubsection{Sample Input 1}\label{trai-dong-2024-f-sample-input-1}

\begin{tcolorbox}
\begin{verbatim}
3 15
7 5 9
10 8 6
\end{verbatim}
\end{tcolorbox}

\subsubsection{Sample Output 1}\label{trai-dong-2024-f-sample-output-1}

\begin{tcolorbox}
\begin{verbatim}
7 14 9
\end{verbatim}
\end{tcolorbox}

\subsubsection{Sample Input 2}\label{trai-dong-2024-f-sample-input-2}

\begin{tcolorbox}
\begin{verbatim}
2 2
1 2
2 1
\end{verbatim}
\end{tcolorbox}

\subsubsection{Sample Output 2}\label{trai-dong-2024-f-sample-output-2}

\begin{tcolorbox}
\begin{verbatim}
1 2
\end{verbatim}
\end{tcolorbox}
