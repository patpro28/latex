\section{Tạo dãy xor}\label{trai-dong-2024-i}

\subtitle{Time limit: \textbf{1 giây} \hfill Memory limit: \textbf{512}Mb}

\subsection{Đề bài}\label{trai-dong-2024-i-de-bai}

Cho hai dãy số nguyên dương $a_1, a_2, ..., a_n$ và
$b_1, b_2, ..., b_n$, ta tạo ra dãy $c_k = a_i\oplus b_j$
($\oplus$ là phép xor hai số nguyên). Hãy cho biết nếu sắp xếp dãy
$c$ theo giá trị tăng dần thì $c_x$ bằng bao nhiêu.

\subsection{Dữ liệu vào}\label{trai-dong-2024-i-du-lieu-vao}

Dòng đầu chứa số $T$ là số lượng test con, với mỗi test con bao gồm:

\begin{itemize}
\tightlist
\item
  Dòng đầu chứa số $n, x$ $(1\le n\le 2\cdot 10^5; 1\le x\le n^2)$
\item
  Dòng thứ hai chứa $n$ số $a_i$ $(1\le a_i\le 2^{60})$
\item
  Dòng thứ ba chứa $n$ số $b_j$ $(1\le b_j\le 2^{60})$
\end{itemize}

\subsection{Dữ liệu ra}\label{trai-dong-2024-i-du-lieu-ra}

Với mỗi test ghi ra giá trị $c_x$ trên một dòng.

\subsection{Tính điểm}\label{trai-dong-2024-i-tinh-diem}

\begin{enumerate}
\def\labelenumi{\arabic{enumi}.}
\tightlist
\item
  $(15p)$: $S_n\le 1000$
\item
  $(20p)$: $a_i, b_j\le 2^{20}; x = 1$ hoặc $x = n^2$
\item
  $(25p)$: $S_n\le 5\cdot 10^4; a_i, b_j\le 2^{20}$
\item
  $(40p)$: không có ràng buộc gì thêm.
\end{enumerate}

\subsubsection{Sample Input 1}\label{trai-dong-2024-i-sample-input-1}

\begin{tcolorbox}
\begin{verbatim}
2
3 5
3 6 9
12 7 13
5 25
3 6 20 5 9
12 3 19 7 13
\end{verbatim}
\end{tcolorbox}

\subsubsection{Sample Output 1}\label{trai-dong-2024-i-sample-output-1}

\begin{tcolorbox}
\begin{verbatim}
10
26
\end{verbatim}
\end{tcolorbox}
