\section{Phủ đoạn}\label{j.-phux1ee7-ux111oux1ea1n}
\subtitle{Time limit: \textbf{1 giây} \hfill Memory limit: \textbf{256}Mb}

\subsubsection{Đề bài}\label{ux111ux1ec1-buxe0i-9}

Trên trục $Ox$ xét một đoạn $[l, r]$ và tập đoạn thẳng $S$ ban đầu
rỗng. Thực hiện $n$ thao tác, mỗi thao tác thuộc một trong hai loại:

\begin{enumerate}
\def\labelenumi{\arabic{enumi}.}
\tightlist
\item
  Thêm một đoạn $[s, t]$ có chỉ số là $idx$ vào tập $S$.
\item
  Loại bỏ đoạn có chỉ số $idx$ khỏi tập $S$.
\end{enumerate}

Sau mỗi thao tác, hãy cho biết có tồn tại một đoạn nào trong tập $S$
mà phủ hoàn toàn đoạn $[l, r]$ hay không. Nếu không, cho biết có hai
đoạn nào trong tập $S$ mà có thể hợp lại thành một đoạn phủ hoàn toàn
đoạn $[l, r]$ hay không?

Một đoạn $[s, t]$ được gọi là \textbf{phủ hoàn toàn} đoạn $[l, r]$
nếu $s \leq l$ và $t \geq r$.

Hai đoạn $[s_1, t_1]$ và $[s_2, t_2]$ có thể \textbf{hợp lại} nếu
$max(s_1, s_2) \leq min(t_1, t_2)$ và khi hợp lại thành đoạn
$[min(s_1, s_2), max(t_1, t_2)]$ thì đoạn này phủ hoàn toàn
$[l, r]$.

\subsubsection{Dữ liệu vào}\label{dux1eef-liux1ec7u-vuxe0o-9}

\begin{itemize}
\tightlist
\item
  Dòng đầu chứa ba số nguyên $l, r, n$
  $(1 \leq l < r \leq 10^9; 1 \leq n \leq 2 \cdot 10^5)$.
\item
  $n$ dòng tiếp theo, mỗi dòng mô tả một trong hai thao tác:

  \begin{itemize}
  \tightlist
  \item
    Nếu là thao tác thêm đoạn, dòng bắt đầu là ký tự \texttt{+} sau đó
    là ba số nguyên $idx, s, t$
    $(1 \leq idx \leq 10^6; 0 \leq s < t \leq 10^9)$. Dữ liệu đảm bảo
    chỉ số của mỗi đoạn là đôi một phân biệt.
  \item
    Nếu là thao tác xóa đoạn, dòng bắt đầu là ký tự \texttt{-} sau đó là
    một số nguyên $idx$ $(1 \leq idx \leq 10^6)$. Dữ liệu đảm bảo
    đoạn có chỉ số $idx$ đã tồn tại trong tập $S$.
  \end{itemize}
\end{itemize}

\subsubsection{Dữ liệu ra}\label{dux1eef-liux1ec7u-ra-9}

\begin{itemize}
\tightlist
\item
  Sau mỗi thao tác, ghi kết quả trên một dòng:

  \begin{itemize}
  \tightlist
  \item
    In \texttt{1} nếu tồn tại một đoạn phủ hoàn toàn đoạn $[l, r]$.
  \item
    In \texttt{2} nếu tồn tại hai đoạn hợp lại được với nhau để phủ hoàn
    toàn đoạn $[l, r]$.
  \item
    Ngược lại, in ra \texttt{-1}.
  \end{itemize}
\end{itemize}

\subsubsection{Giới hạn}\label{giux1edbi-hux1ea1n-8}

\begin{itemize}
\tightlist
\item
  $1 \leq n \leq 2 \cdot 10^5$
\item
  $1 \leq idx \leq 10^6$
\item
  $0 \leq s < t \leq 10^9$
\end{itemize}

\subsubsection{Subtasks}\label{subtasks-6}

\begin{enumerate}
\def\labelenumi{\arabic{enumi}.}
\tightlist
\item
  (30 điểm): $n \leq 200$
\item
  (30 điểm): $n \leq 2000$
\item
  (40 điểm): Không có ràng buộc gì thêm.
\end{enumerate}

\subsubsection{Sample Input}\label{sample-input-9}

\begin{tcolorbox}
\begin{verbatim}
2 6 5
+ 2 4 6
+ 1 2 3
- 2
+ 3 3 7
+ 5 2 6
\end{verbatim}
\end{tcolorbox}

\subsubsection{Sample Output}\label{sample-output-9}

\begin{tcolorbox}
\begin{verbatim}
-1
-1
-1
2
1
\end{verbatim}
\end{tcolorbox}
