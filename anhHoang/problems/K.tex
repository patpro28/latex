\section{Cặp số hài hoà}\label{cux1eb7p-sux1ed1-huxe0i-houxe0}
\subtitle{Time limit: \textbf{2 giây} \hfill Memory limit: \textbf{1024}Mb}

\subsubsection{Đề bài}\label{ux111ux1ec1-buxe0i-10}

Một tập độc lập cực đại trong đồ thị là tập có kích thước lớn nhất chứa
các đỉnh trong đồ thị sao cho không có hai đỉnh thuộc tập này có cạnh
nối đến nhau.

Bạn được cho một cây đồ thị gồm $n$ đỉnh, đánh số từ $1$ đến $n$.
Một cặp đỉnh $(x, y)$ được gọi là hài hoà nếu ta thêm cạnh nối giữa
hai đỉnh $x$ và $y$ (cạnh này có thể là một cạnh nằm trên cây hoặc
không), sao cho kích thước của tập độc lập cực đại là không đổi.

Nhiệm vụ của bạn là đếm số lượng cặp đỉnh hài hoà.

\subsubsection{Dữ liệu vào}\label{dux1eef-liux1ec7u-vuxe0o-10}

\begin{itemize}
\tightlist
\item
  Dòng đầu tiên chứa số nguyên $n$ $(2 \leq n \leq 2.5 \cdot 10^5)$.
\item
  Trong $n-1$ dòng tiếp theo, mỗi dòng chứa hai số nguyên $u_i, v_i$
  mô tả các cạnh của cây.
\end{itemize}

\subsubsection{Dữ liệu ra}\label{dux1eef-liux1ec7u-ra-10}

\begin{itemize}
\tightlist
\item
  Ghi trên một dòng một số nguyên duy nhất là số lượng cặp đỉnh hài hoà.
\end{itemize}

\subsubsection{Giới hạn}\label{giux1edbi-hux1ea1n-9}

\begin{itemize}
\tightlist
\item
  $2 \leq n \leq 2.5 \cdot 10^5$
\end{itemize}

\subsubsection{Subtasks (có thể có)}\label{subtasks-cuxf3-thux1ec3-cuxf3}

\begin{enumerate}
\def\labelenumi{\arabic{enumi}.}
\tightlist
\item
  (15 điểm): $n \leq 18$
\item
  (25 điểm): $n \leq 400$
\item
  (30 điểm): $n \leq 2000$
\item
  (30 điểm): Không có ràng buộc gì thêm.
\end{enumerate}

\subsubsection{Sample Input}\label{sample-input-10}

\begin{tcolorbox}
\begin{verbatim}
4
1 2
1 3
1 4
\end{verbatim}
\end{tcolorbox}

\subsubsection{Sample Output}\label{sample-output-10}

\begin{tcolorbox}
\begin{verbatim}
3
\end{verbatim}
\end{tcolorbox}

\subsubsection{Giải thích}\label{giux1ea3i-thuxedch-7}

Có ba cặp đỉnh hài hoà là $(1, 2)$, $(1, 3)$ và $(1, 4)$. Do tập
độc lập cực đại là $(2, 3, 4)$. Nếu thêm cạnh nối giữa đỉnh $1$ và
một trong ba đỉnh còn lại, ta vẫn có tập độc lập cực đại là
$(2, 3, 4)$.
