\section{Cóc nhảy}\label{f.-cuxf3c-nhux1ea3y}
\subtitle{Time limit: \textbf{1 giây} \hfill Memory limit: \textbf{256}Mb}

\subsubsection{Đề bài}\label{ux111ux1ec1-buxe0i-5}

Cho $n$ điểm trên trục $Ox$, một con cóc sẽ nhảy qua hết $n$ điểm
theo quy tắc sau: nếu nó đang ở điểm $p$ thì nó sẽ nhảy sang điểm gần
nhất mà nó chưa đứng ở điểm đó. Nếu có hai điểm có cùng khoảng cách nhỏ
nhất thì nó sẽ nhảy sang điểm có tọa độ nhỏ hơn.

Hãy cho biết điểm cuối cùng mà con cóc nhảy đến nếu ban đầu nó đứng ở
điểm $s$ với mọi $0 \leq s < n$.

\subsubsection{Dữ liệu vào}\label{dux1eef-liux1ec7u-vuxe0o-5}

\begin{itemize}
\tightlist
\item
  Dòng đầu chứa số nguyên $n$ $(2 \leq n \leq 10^6)$.
\item
  Dòng thứ hai chứa $n$ số $x_i$
  $(1 \leq x_i \leq 10^{12}; x_i < x_{i+1})$, là tọa độ của $n$
  điểm.
\end{itemize}

\subsubsection{Dữ liệu ra}\label{dux1eef-liux1ec7u-ra-5}

\begin{itemize}
\tightlist
\item
  Ghi ra $n$ số tương ứng với kết quả của $s$ từ $0$ đến $n-1$.
\end{itemize}

\subsubsection{Giới hạn}\label{giux1edbi-hux1ea1n-4}

\begin{itemize}
\tightlist
\item
  $2 \leq n \leq 10^6$
\item
  $1 \leq x_i \leq 10^{12}$
\item
  Các tọa độ $x_i$ khác nhau và tăng dần.
\end{itemize}

\subsubsection{Subtasks}\label{subtasks-2}

\begin{enumerate}
\def\labelenumi{\arabic{enumi}.}
\tightlist
\item
  (30 điểm): $n \leq 1000$
\item
  (30 điểm): $n \leq 10^5$
\item
  (40 điểm): Không có ràng buộc gì thêm.
\end{enumerate}

\subsubsection{Sample Input}\label{sample-input-5}

\begin{tcolorbox}
\begin{verbatim}
5
2 5 6 8 11
\end{verbatim}
\end{tcolorbox}

\subsubsection{Sample Output}\label{sample-output-5}

\begin{tcolorbox}
\begin{verbatim}
4 0 4 4 0
\end{verbatim}
\end{tcolorbox}

\subsubsection{Giải thích}\label{giux1ea3i-thuxedch-4}

\textbf{Dạng nhảy của con cóc:}

\begin{itemize}
\tightlist
\item
  Nếu bắt đầu ở $s = 0$: Con cóc lần lượt nhảy từ điểm
  $2 \to 5 \to 6 \to 8 \to 11$. Kết quả cuối cùng là điểm $4$ (tọa
  độ $11$).
\item
  Nếu bắt đầu ở $s = 1$: Con cóc lần lượt nhảy từ điểm
  $5 \to 6 \to 8 \to 11 \to 2$. Kết quả cuối cùng là điểm $0$ (tọa
  độ $2$).
\item
  Nếu bắt đầu ở $s = 2$: Con cóc lần lượt nhảy từ điểm
  $6 \to 5 \to 2 \to 8 \to 11$. Kết quả cuối cùng là điểm $4$ (tọa
  độ $11$).
\item
  Nếu bắt đầu ở $s = 3$: Con cóc lần lượt nhảy từ điểm
  $8 \to 6 \to 5 \to 2 \to 11$. Kết quả cuối cùng là điểm $4$ (tọa
  độ $11$).
\item
  Nếu bắt đầu ở $s = 4$: Con cóc lần lượt nhảy từ điểm
  $11 \to 8 \to 6 \to 5 \to 2$. Kết quả cuối cùng là điểm $0$ (tọa
  độ $2$).
\end{itemize}
