\subsection{Euler's totient function}\label{eulertotientfunction}

\subsubsection{Định nghĩa}\label{etf-defination}

Hàm số Euler $\phi(n)$ là số lượng số nguyên dương nhỏ hơn hoặc bằng $n$ và nguyên tố cùng nhau với $n$.

$$\phi(n) = |\{k \in \mathbb{N} | 1 \leq k \leq n, \gcd(k, n) = 1\}|$$

Ví dụ: $\phi(8) = 4$ vì có 4 số nguyên dương nhỏ hơn hoặc bằng 8 và nguyên tố cùng nhau với 8 là 1, 3, 5, 7.

Dưới đây là một số giá trị của hàm số Euler $\phi(n)$ của một số số đầu tiên:

\begin{table}[h!]
\begin{tabular}{|c|c|c|c|c|c|c|c|c|c|c|c|c|c|c|c|c|c|c|c|c|c|c|c}
\hline
$n$ & 1 & 2 & 3 & 4 & 5 & 6 & 7 & 8 & 9 & 10 & 11 & 12 & 13 & 14 & 15 & 16 & 17 & 18 & 19 & 20 & 21 & 22 \\
\hline
$\phi(n)$ & 1 & 1 & 2 & 2 & 4 & 2 & 6 & 4 & 6 & 4 & 10 & 4 & 12 & 6 & 8 & 8 & 16 & 6 & 18 & 8 & 12 & 10 \\
\hline
\end{tabular}
\end{table}

\subsubsection{Công thức tính}\label{etf-formula}

Nếu $n$ có dạng phân tích thừa số nguyên tố là $n = p_1^{a_1}p_2^{a_2}\ldots p_k^{a_k}$ thì hàm số Euler $\phi(n)$ có thể tính bằng công thức sau:

$$\phi(n) = n \cdot \left(1 - \frac{1}{p_1}\right) \cdot \left(1 - \frac{1}{p_2}\right) \ldots \left(1 - \frac{1}{p_k}\right)$$

\subsubsection{Thuật toán tính}\label{etf-algorithm}

\begin{tcolorbox}
\begin{minted}{cpp}
int phi(int n) {
    int result = n;
    for (int i = 2; i * i <= n; i++) {
        if (n % i == 0) {
            while (n % i == 0) {
                n /= i;
            }
            result -= result / i;
        }
    }
    if (n > 1) {
        result -= result / n;
    }
    return result;
}
\end{minted}
\end{tcolorbox}

\subsubsection{Tính chất}\label{etf-properties}

\begin{enumerate}
    \item Hàm nhân tính: $\phi(mn) = \phi(m) \cdot \phi(n)$ nếu $\gcd(m, n) = 1$.
    \item Trường hợp $m$ và $n$ không nguyên tố cùng nhau: 
        $$\phi(mn) = \phi(m) \cdot \phi(n) \cdot \frac{d}{\phi(d)}$$ với $d = \gcd(m, n)$.

        Trường hợp đặc biệt:
        \begin{align}
            \phi(2m) &= \begin{cases}
                \phi(m) & \text{nếu $m$ lẻ} \\
                2\cdot\phi(m) & \text{nếu $m$ chẵn}
            \end{cases} \\
            \phi(n^m) &= n^{m - 1} \cdot \phi(n)
        \end{align}
    \item Nếu $p$ là số nguyên tố và $k$ là số nguyên dương, thì: $$\phi(p^k) = p^k - p^{k - 1}$$
    \item Mối quan hệ với các ước của $n$:
        $$\sum_{d|n} \phi(d) = n$$
        Mở rộng với $\gcd$:
        $$\sum_{d|a \text{ and } d|b} \phi(d) = \gcd(a, b)$$
    \item Nếu $a$ là ước của $b$ thì $\phi(a)$ cũng là ước của $\phi(b)$.
    \item Số lượng các số nguyên $k$ thoả mãn $\gcd(k, n) = d$ là:
        $$\sum_{k=1}^{n} [\gcd(k, n) = d] = \phi\Big(\frac{n}{d}\Big)$$

        Ứng dụng:

        \begin{align}
            \sum_{k=1}^{n} \gcd(k, n) &= \sum_{d|n} d \cdot \phi\Big(\frac{n}{d}\Big) \\
            \sum_{k=1}^{n} x^{\gcd(k, n)} &= \sum_{d|n} x^d \cdot \phi\Big(\frac{n}{d}\Big) \\
            \sum_{k=1}^{n} \frac{1}{\gcd(k, n)} &= \sum_{d|n} \frac{1}{d} \cdot \phi\Big(\frac{n}{d}\Big) = \frac{1}{n} \sum_{d|n} d \cdot \phi(d) \\
            \sum_{k=1}^{n} \frac{k}{\gcd(k, n)} &= \frac{n}{2} \sum_{d|n} \frac{1}{d} \cdot \phi(\frac{n}{d}) = \frac{1}{2} \sum_{d|n} d \cdot \phi(d) \\
            \sum_{i=1}^{n}\sum_{j=1}^{n} \gcd(i, j) &= \sum_{d=1}^{n} \phi(d) \cdot \left\lfloor\frac{n}{d}\right\rfloor^2 \\
            \sum_{i=1}^{n} \text{lcm}(i, n) &= \frac{n}{2}\Big(\sum_{d|n}(\phi(d)\cdot d) + 1\Big)
        \end{align}
\end{enumerate}